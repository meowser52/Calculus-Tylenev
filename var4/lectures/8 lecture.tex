\begin{proposition}
    Пусть $\{ x_{n} \}$~---~числовая последовательность. Множество её частичных пределов образуют замкнутое множество.
    
    То есть: 
    $\{ x_{n} \}$ - произвольная числовая последовательность, то $(PL(\{ x_{n} \}))$ - замкнутое множетво.
    
\[\{ x_{n} \} \cup PL(\{ x_{n} \}) = \text{cl}(\{ x_{n} \})
\]
\end{proposition}

\begin{problem}
    Существует ли последовательность, у которой множество частичных пределов несчётно?
\end{problem}

\begin{solution}
    Да. Например, рациональные точки на числовой прямой: $\Q \{ r_{n} \}_{n=1}^{\infty}$ множество пределов последовательности $PL(\{ r_{n} \}) = \R.$ Это следует из того, что в любой $U_{\epsilon}$ любой точки содержится как бесконечно много рациональных, так и иррациональных точек.
\end{solution}

\newpage
\section{Предел функций}

\subsection{Классические определения предела}

\begin{definition}
    Под \textit{функцией}, если не оговорено обратное, понимаем однозначное отображение $f: E \mapsto \R,$ где $E \subset \R, E\neq \varnothing.$
\end{definition}
\begin{definition}
    Пусть $\delta > 0, \  x_{0} \in \overline{\R}$. Тогда \textit{проколотой дельта-окрестностью} точки $x_{0}$ называется множество $\mathring{U}_{\delta} (x_{0}) := U_{\delta} (x_{0}) \backslash \{ x_{0} \}$
\end{definition}
\begin{note} Если $x_0 = \pm \infty$ или $x_0 = \infty$, то проколотая окрестность совпадает с непроколотой. Если же $x_{0} \in \R$, то $\mathring{U}_{\delta} = (x_0 -\delta, x_0 ) \cup (x_0, x_0 + \delta)$.
\end{note}
\begin{definition}
    (\hypertarget{def4.3}{По Коши/в терминах окрестностей}) Пусть $x_0 \in \Hat{\R}$ и пусть $A \in \Hat{\R}. $ Пусть $f$: $\mathring{U}_{\delta_{0}} \mapsto \R.$ Будем говорить, что $A$~---~ предел функции $f$ в точке $x_{0}$ и записывать:
    
\[
\left[
\begin{gathered}
    \lim\limits_{x\to x_0} f(x) = A
    \hfill 
    \\ 
    f(x) \to A, x \to x_0
\end{gathered}
\right.
\Leftrightarrow
\forall \epsilon > 0 \  \exists \  \delta(\epsilon) \in (0; \delta_0]: \ \forall x \in \mathring{U}_{\delta}(x_0) \  \hookrightarrow \ f(x) \in U_\epsilon(A)
\]
\begin{center}
    \centering
    \begin{tabular}{cc}
    \begin{tikzpicture}[scale=0.8, decoration=curveto]
        \begin{axis}
            [
            samples=500, 
            axis lines = middle, 
            xlabel = {$x$},
            ylabel = {$y$},
            xtick ={4, 5},
            xticklabels={$x_0$, $ $},
            ytick={1.33, 4},
            yticklabels={$A$, $ $},
            xmax = 5,
            ymax = 2
            ]
            \addplot[]{x^(1/2)+cos(deg(x))};
        \end{axis}
            \draw[] (6.70, 3.30) circle (1pt);
            \draw[dotted] (0, 2.8) -- (8, 2.8);
            \draw[<->] (5, 2.8) -- (5, 3.8);
            \draw[dotted] (0, 3.8) -- (8, 3.8);
            \node[left] at (5, 3.25) {$2\epsilon$};
            \draw[dotted] (6.3, 0) -- (6.3, 7);
            \draw[<->] (6.3, 1) -- (7.2, 1);
            \draw[dotted] (7.2, 0) -- (7.2, 7);
            \node[above] at (6.65, 1) {$2\delta$};

            \node[below] at(2.5, 0) {$A\in \R, x_0 \in \R$};
          
\end{tikzpicture}
%\caption{$A\in \R, x_0 \in \R$}
&
    \begin{tikzpicture}[scale=0.8, decoration=curveto]
        \draw[->, thin] (0,0) -- (8,0)
        node[below] {$x$}; % Ox
        \draw[->, thin] (0,0) -- (0, 7)
        node[left] {$y$}; % Oy
            \draw (0, 0) arc (-90:0:5.5 and 7) -- cycle;
            \draw[white, line width = .05 cm] (0,0) -- (5.5, 7);
        \draw[dotted] (0, 5) -- (8, 5);
        \node[left] at (0, 5) {$1 / \epsilon$};
        \draw[dotted] (5, 0) -- (5, 7);
        \draw[<->] (5, 1) -- (6, 1);
        \draw[dotted] (6, 0) -- (6, 7);
        \draw[dotted] (5.5, 0) -- (5.5, 7);
        \node[above] at (5.5, 1) {$2\delta$};

            \node[below] at (5.5, 0) {$x_0$};
            \filldraw[pattern=north east lines] (5,5) rectangle (6,7);
            \draw[white, line width = .03 cm] (5,7) -- (6,7);
          \node[below] at(2.5, 0) {$A=+\infty, x_0 \in \R$};
\end{tikzpicture}
%\caption{$A=+\infty, x_0 \in \R$}
    \end{tabular}
\end{center}

\end{definition}

\begin{definition}
    \textit{Последовательностью Гейне} в точке $x_{0} \in \Hat{\R}$ называется такая числовая последовательность $\{ x_{n} \} \subset \R$, что выполнено два условия:
    \begin{enumerate}
        \item $\lim\limits_{n\to \infty} x_{n} = x_{0};$
        \item $x_{n} \neq x_{0} \  \forall n \in \N.$
    \end{enumerate}
\end{definition}
\begin{definition}
    \hypertarget{def4.5}{(По Гейне/в терминах последовательностей)} Пусть $x_{0} \in \Hat{\R}$ и $A \in \Hat{\R}$. Пусть $f$: $\mathring{U_{\delta_{0}}} (x_{0}) \mapsto \R.$ Будем говорить, что $f$ имеет предел в точке $x_{0}$ равный $A$, если для любой последовательности Гейне $\{ x_{n} \} \subset \mathring{U}_{\delta_{0}} (x_{0})$ в точке $x_{0}$ $\exists \lim\limits_{n\to \infty} f (x_{n}) = A.$ Пишем:
    $
\left[
\begin{gathered}
    \lim\limits_{x\to x_0} f(x) = A
    \hfill 
    \\ 
    f(x) \to A, x \to x_0
\end{gathered}
\right.$
\end{definition}
\begin{theorem}
    \hypertarget{thm4.1}{(Эквивалентность определений по Коши и по Гейне)} Пусть $x_{0} \in \Hat{\R}$, $A \in \Hat{\R}$. Пусть $f$: $\mathring{U}_{\delta_{0}} (x_{0}) \mapsto \R$. Тогда $\lim\limits_{x\to x_{0}} f (x) = A$ (по Коши) $\Leftrightarrow \lim\limits_{x\to x_{0}} f (x) = A$ (по Гейне).
\end{theorem}
\begin{proof}
    \underline{Шаг 1.} Докажем сначала, что из Коши следует Гейне. Распишем определение:
    $$ \lim\limits_{x\to x_{0}} f (x) = A \Leftrightarrow \forall \epsilon > 0 \ \exists \delta (\epsilon) \in (0; \delta_{0}]: \forall x \in \mathring{U}_{\delta (\epsilon)} (x_{0}) \hookrightarrow f (x) \in U_{\epsilon} (A). \quad (1)$$

    Возьмём произвольную последовательность Гейне $\{ x_{n} \} \subset \mathring{U}_{\delta_{0}} (x_{0})$ в точке $x_{0}$. Первый пункт определения выполнен автоматически (так как мы берём из проколотой дельта-окрестности). Запишем определение предела последовательности:
    $$ \lim\limits_{n\to \infty} x_{n} = x_{0} \Rightarrow \forall \delta \in (0; \delta_{0}] \ \exists N (\delta) \in \N: \forall n \geq N (\delta) \hookrightarrow x_{n} \in \mathring{U}_{\delta} (x_{0}).$$
    В частности, если фиксирован произвольный $\epsilon > 0$, то $\forall n \geq N (\delta (\epsilon) ) \hookrightarrow x_{n} \in \mathring{U}_{\delta (\epsilon)} (x_{0}).$ Положим $\Tilde{N} (\epsilon) := N (\delta (\epsilon)).$ Из $(1) \implies \forall n \geq \Tilde{N} (\epsilon) \hookrightarrow f (x_{n}) \in U_{\epsilon} (A).$ Но поскольку $\epsilon > 0$ был выбран произвольно, то
    $$ \forall \epsilon > 0 \  \exists \Tilde{N} (\epsilon) = N (\delta (\epsilon)) \in \N: \forall n \geq \Tilde{N} (\epsilon) \hookrightarrow f (x_{n}) \in U_{\epsilon} (A) \Rightarrow \lim\limits_{n\to \infty} f(x_{n}) = A.$$

    Но так как $\{ x_{n} \}$~---~последовательность Гейне в точке $x_{0}$ была выбрана произвольно, то  $\exists \lim\limits_{n\to \infty} f (x_{n}) = A$ по Гейне.

    \underline{Шаг 2.} Докажем, что из Гейне следует Коши. Предположим, что $\exists \lim\limits_{x\to x_{0}} f (x) = A$ по Гейне, но не по Коши. Запишем отрицание к Коши:

    $$ \exists \epsilon > 0: \forall \delta \in (0; \delta_{0}] \   \exists x \in \mathring{U}_{\delta} (x_{0}) \hookrightarrow f (x) \notin U_{\epsilon} (A).$$

    Раз это верно для любого $\delta$, то возьмём $\delta = \frac{\delta_{0}}{n}$. Получаем:

    $$ \exists \epsilon > 0: \forall n \in \N \  \exists x_{n} \in \mathring{U}_{\frac{\delta_{0}}{n}} (x_{0}): f (x_{n}) \notin U_{\epsilon} (A).$$

    (Примечание на лекции: здесь в неявной форме используется аксиома выбора) Получется последовательность $\{ x_{n} \} \subset \mathring{U}_{\delta_{0}} (x_{0})$. А по построению $\lim\limits_{n\to \infty} x_{n} = x_{0} \Rightarrow \{ x_{n} \}$~---~последовательность Гейне в точке $x_{0}$. Но $\forall n \in \N \  f (x_{n}) \notin U_{\epsilon} (A) \Rightarrow \lim\limits_{n\to \infty} f (x_{n}) \neq A$. Получаем противоречие с существованием предела по Гейне. Значит предположение было неверно и из Гейне следует Коши.
\end{proof}

\subsection{Предел по множеству}

\begin{definition}
    Пусть $E \subset \R$, $E \neq \varnothing$, а $x_{0} \in \Hat{\R}$. Будем говорить, что $x_{0}$~---~\textit{предельная точка} множества $E$, если
    $$ \forall \delta > 0 \  \mathring{U}_{\delta} (x_{0}) \cap E \neq \varnothing.$$
\end{definition}
\begin{lemma}
    Пусть $E \subset \R$, $E \neq \varnothing$. $x_{0} \in \Hat{\R}$~---~\textit{предельная точка} $\Leftrightarrow \exists \{ x_{n} \} \in E \backslash \{ x_{0} \}$: $\exists \lim\limits_{n\to \infty} x_{n} = x_{0}$. Любую такую последовательность будем называть последовательностью Гейне в точке $x_{0}$ для множества $E$.
\end{lemma}
\begin{definition}
    (Предел по множеству) Пусть $A \in \Hat{\R}$, $x_{0} \in \Hat{\R}$. Пусть $f$: $E \mapsto \R$, $E \neq \varnothing$ и $x_{0}$~---~предельная точка множества $E$. Будем говорить, что $A$~---~предел $f$ по множеству $E$ при $x \to x_{0}$ и записывать это $\lim\limits_{\underset{x \in E}{x\to x_{0}}} f (x) = A$, если
    $$ \forall \epsilon > 0 \  \exists \delta (\epsilon) > 0: \forall x \in E \cap  \mathring{U}_{\delta (\epsilon)} (x_{0}) \hookrightarrow f (x) \in U_{\epsilon} (A) \text{~---~по Коши.}$$
    $$ \forall \text{ последовательности Гейне }\{ x_{n}\} \subset E \  \text{в точке } x_{0} \hookrightarrow \lim\limits_{n\to \infty} f (x_{n}) = A.$$
\end{definition}
\begin{theorem}
    Определения предела по множеству в терминах Коши и Гейне эквивалентны.
\end{theorem}
\begin{proof}
    Абсолютно аналогично доказательству \hyperlink{thm4.1}{теоремы 4.1}.
\end{proof}
\begin{lemma}
    \hypertarget{lemm4.2}{Пусть} $E_{1}$, $E_{2} \subset \R$. Пусть $A \in \Hat{\R}$ и $x_{0} \in \Hat{\R}$. Пусть $x_{0}$~---~предельная точка и для $E_{1}$, и для $E_{2}$. Тогда следующие условия эквивалентны:
    $$ (1) \  \lim\limits_{\underset{x \in E_{1} \cup E_{2}}{x\to x_{0}}} f (x) = A \Leftrightarrow 
    \begin{cases}
            \lim\limits_{\underset{x \in E_{1}}{x\to x_{0}}} f (x) = A\\
            \lim\limits_{\underset{x \in E_{2}}{x\to x_{0}}} f (x) = A& 
        \end{cases}
        \quad (2)
    $$
\end{lemma}
\begin{proof}
    Докажем сначала $(1) \to (2)$. Распишем $(1)$ по определению:

    $$\forall \epsilon > 0 \  \exists \delta (\epsilon) > 0: \forall x \in (E_{1} \cup E_{2}) \cap \mathring{U}_{\delta (\epsilon)} (x_{0}) \hookrightarrow f (x) \in U_{\epsilon} (A).$$

    Тогда
    $
    \begin{cases}
        \forall \epsilon > 0 \  \exists \delta (\epsilon) > 0: \forall x \in E_{1} \cap \mathring{U}_{\delta (\epsilon)} (x_{0}) \hookrightarrow f (x) \in U_{\epsilon} (A) \\
        \forall \epsilon > 0 \  \exists \delta (\epsilon) > 0: \forall x \in E_{2} \cap \mathring{U}_{\delta (\epsilon)} (x_{0}) \hookrightarrow f (x) \in U_{\epsilon} (A) &
    \end{cases}
    \Rightarrow
    \begin{cases}
            \lim\limits_{\underset{x \in E_{1}}{x\to x_{0}}} f (x) = A\\
            \lim\limits_{\underset{x \in E_{2}}{x\to x_{0}}} f (x) = A& 
        \end{cases}
    $

    Теперь докажем $(2) \to (1)$. То есть выполнена система
    $$
    \begin{cases}
            \lim\limits_{\underset{x \in E_{1}}{x\to x_{0}}} f (x) = A\\
            \lim\limits_{\underset{x \in E_{2}}{x\to x_{0}}} f (x) = A& 
        \end{cases}
    \Rightarrow
    \begin{cases}
        \forall \epsilon > 0 \  \exists \delta_{1} (\epsilon) > 0: \forall x \in E_{1} \cap \mathring{U}_{\delta_{1} (\epsilon)} (x_{0}) \hookrightarrow f (x) \in U_{\epsilon} (A) \\
        \forall \epsilon > 0 \  \exists \delta_{2} (\epsilon) > 0: \forall x \in E_{2} \cap \mathring{U}_{\delta_{2} (\epsilon)} (x_{0}) \hookrightarrow f (x) \in U_{\epsilon} (A) &
    \end{cases}
    $$
    Получается
    $$ \forall \epsilon > 0 \  \exists \delta (\epsilon) = min \{ \delta_{1} (\epsilon), \delta_{2} (\epsilon)\} > 0: \forall x \in (E_{1} \cup E_{2}) \cap \mathring{U}_{\delta (\epsilon)} (x_{0}) = (E_{1} \cap \mathring{U}_{\delta} (\epsilon)) \cup (E_{2} \cap \mathring{U}_{\delta} (\epsilon)) \hookrightarrow$$
    $$\hookrightarrow f (x) \in U_{\epsilon} (A).$$
\end{proof}
\begin{definition}
    \textit{Функция Дирихле} $f = 
    \begin{cases}
        0, \ x \in \R \backslash \Q \\
        1, \ x \in \Q &
    \end{cases}
    $
\end{definition}
\begin{proposition}
    Функция Дирихле не имеет предела ни в какой точке.
\end{proposition}
\begin{proof}
    Пусть $x_{0} \in \Hat{\R}.$
    
    $\lim\limits_{\underset{x \in \Q}{x\to x_{0}}} f (x) = 1$, а $\lim\limits_{\underset{x \in \R \backslash \Q}{x\to x_{0}}} f (x) = 0.$ По только что доказанной нами \hyperlink{lemm4.2}{лемме} $\nexists \lim\limits_{\underset{x \in \R}{x\to x_{0}}} f (x) \Rightarrow \nexists \lim\limits_{x\to x_{0}} f (x).$
\end{proof}
\subsection{Критерий Коши для функций}
\begin{lemma}
    \hypertarget{lemm4.3}{Пусть} $f$: $E \mapsto \R$. Пусть $x_{0}$~---~предельная точка множества $E$. Пусть для любой последовательности Гейне $\{ x_{n} \} \subset E$ в точке $x_{0}$ $\exists \lim\limits_{n\to \infty} f (x_{n}) = A$, $A \in \Hat{\R}$. Тогда $A$ не зависит от выбора $\{ x_{n} \}$. То есть $A$ одинаково при выборе любого $\{ x_{n}\}$.
\end{lemma}
\begin{proof}
    Пусть есть две последовательности Гейне $\{ x_{n} \}$, $\{ y_{n} \} \subset E$ в точке $x_{0}$.

    Положим $\{ z_{n} \} =
    \begin{cases}
        x_{k}, \ n = 2k \\
        y_{k}, \ n = 2k - 1&
    \end{cases}  
    $

    Тогда $\{ z_{n} \} \subset E \text{ и $\{ z_{n} \}$~---~последовательность Гейне в точке } x_{0}.$ 
    
    По условию $\exists \lim\limits_{n\to \infty} f (z_{n}) = A.$ Но тогда $\{ f (x_{k}) \}$ и $\{ f (y_{k}) \}$~---~подпоследовательности последовательности $\{ f (z_{n}) \}^{\infty}_{n = 1}$. Тогда $\lim\limits_{k\to \infty} f (x_{k}) = \lim\limits_{k\to \infty} f (y_{k}) = \lim\limits_{n\to \infty} f (z_{n}) = A.$ Но мы взяли две произвольные последовательности Гейне $\Rightarrow A$ один и тот же.
\end{proof}