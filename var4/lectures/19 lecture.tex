\begin{theorem}
	\hypertarget{thrm5.15}{(Формула Тейлора с остаточным членом в форме Лагранжа)}
	Пусть $\exists f^{(n+1)}$ в некоторой $U_{\delta}(x_{0})$. Тогда $\forall x\in \mathring{U}_{\delta}(x_{0})$ справедливо равенство
	
	$$f(x) = \sum\limits_{k = 0}^n \cfrac{f^{(k)}}{k!}(x-x_{0})^{k} + \cfrac{f^{(n+1)}(\xi)}{(n+1)!}(x-x_{0})^{(n+1)}, \ \textrm{где} \ \xi \in (x_{0}, x).$$
\end{theorem}
\begin{proof}
	$\phi_{n+1}(x) = (x-x_{0})^{n+1}$
	
	$$r_{x_{0}}^{n} [f](x) = f(x_{0}) - T_{x_{0}}^{n} [f](x)$$
	
	Можно применить \hyperlink{5.11}{теорему Коши} $n+1$ раз:
	
	$$\cfrac{r_{x_{0}}^{n} [f](x)}{\phi_{n+1}(x)} = \cfrac{\Big(r_{x_{0}}^{n} [f]\Big)^{(n+1)}(\xi_{n+1})}{(n+1)!} $$
	
	Так как $(n+1)$-ая произвдная полинома степени $n$ равна нулю, то справедливо равенство
	
	$$\cfrac{\Big(r_{x_{0}}^{n} [f]\Big)^{(n+1)}(\xi_{n+1})}{(n+1)!} = \cfrac{f^{(n+1)}(\xi_{n+1})}{(n+1)!}, \xi := \xi_{n+1} $$
\end{proof}

\begin{theorem}
	\hypertarget{thrm5.16}{(Теорема о единственности)} Пусть $\exists f^{(n)}(x_{0}) \in \R.$ Тогда, если 
	
	$f(x) = \sum\limits_{k =0}^{n} a_{k}(x-x_{0})^{k} + o\Big((x-x_{0})^{n}\Big), x \to x_{0},$ то $a_{k} = \cfrac{f^{(k)}(x_{0})}{k!}$
\end{theorem}
\begin{note}
	Теорема верна, только если выполняется условие $\exists f^{(n)}(x_{0}) \in \R.$ Может случиться так, что производная не существует, но разложение есть.
\begin{example}
	Задача на дом
\end{example} 
\end{note}
\begin{proof}
	Так как $\exists f^{(n)}(x_{0}) \in \R$, можно воспользоваться \hyperlink{thrm5.14}{формулой Тейлора с остаточным членом в форме Пеано}. Тогда 
	
	$$\begin{cases}
		f(x) = \sum\limits_{k = 0}^{n} a_{k} (x-x_{0})^{k} +  o\Big((x-x_{0})^{n}\Big) , x \to x_{0}, \\
		f(x) = \sum\limits_{k = 0}^{n} \cfrac{f^{(k)}(x_{0})}{k!} (x-x_{0})^{k} +  o\Big((x-x_{0})^{n}\Big) , x \to x_{0},
	\end{cases}$$
	
	$$ \lim\limits_{x\to x_{0}} f(x) =  a_{0} = f(x_{0})$$
	
	$$
	\begin{cases}
		\cfrac{f(x)-f(x_{0})}{x-x_{0}} = \sum\limits_{k=1}^{n} a_{k} (x-x_{0})^{k-1} +o\Big((x-x_{0})^{n-1}\Big) , x \to x_{0} \\
		\cfrac{f(x)-f(x_{0})}{x-x_{0}} = \sum\limits_{k=1}^{n} \cfrac{f^{(k)}(x_{0})}{k!} (x-x_{0})^{k-1} +o\Big((x-x_{0})^{n-1}\Big) , x \to x_{0}
	\end{cases}
	 $$
	 
	 $$ a_{1} = f'(x_{0})$$
	 
	 И так далее $n$ шагов. Получим, что $a_{k} = \cfrac{f^{(k)}(x_{0})}{k!}$
\end{proof}

\begin{theorem}
	\hypertarget{thrm5.17}{(Почленное дифференцрование формулы Тейлора)} Пусть $n\in \N$ и $\exists f^{(n)}(x_{0}).$ Тогда если $f(x) = \sum\limits_{k =0}^{n} a_{k}(x-x_{0})^{k} + o\Big((x-x_{0})^{n}\Big), x \to x_{0},$ то 
	
	$$f'(x) = \sum\limits_{k = 1}^{n} a_{k} \cdot k \cdot (x-x_{0})^{k-1} + o\Big((x-x_{0})^{n-1}\Big), x \to x_{0}.$$
\end{theorem}
\begin{proof}
	В силу \hyperlink{thrm5.16}{теоремы о единственности} $a_{k} = \cfrac{f^{(k)}(x_{0})}{k!}$.
	
	\hyperlink{thrm5.14}{По формуле Тейлора с остаточным членом в форме Пеано}, примененной к функции $f'$, получим, что $f'(x) = \sum\limits_{k = 0}^{n-1} \cfrac{(f')^{(k)}(x_{0})}{k!} (x-x_{0})^{k} + o\Big((x-x_{0})^{n-1}\Big), x \to x_{0}$
	
	$$
	j = k+1
	$$
	$$
	f'(x) = \sum\limits_{j = 1}^{n} \cfrac{f^{(j)}(x_{0})}{(j-1)!} (x-x_{0})^{j-1} + o\Big((x-x_{0})^{n-1}\Big), x \to x_{0}
	$$
	$$
	f'(x) = \sum\limits_{j = 1}^{n} \cfrac{f^{(j)}(x_{0})}{j!} \cdot j \cdot (x-x_{0})^{j-1} + o\Big((x-x_{0})^{n-1}\Big), x \to x_{0}
	$$
	$$
	k = j
	$$
	$$f'(x) = \sum\limits_{k = 1}^{n} a_{k} \cdot k \cdot (x-x_{0})^{k-1} + o\Big((x-x_{0})^{n-1}\Big), x \to x_{0}.$$
\end{proof}

\begin{theorem}
	\hypertarget{thrm5.17}{(Почленное интегрирование формулы Тейлора)} Пусть $\exists f^{(n+1)}(x_{0})$ и 
	
	$f'(x) = \sum\limits_{k = 0}^{n} b_{k}(x-x_{0})^{k} + o \Big((x-x_{0})^{n}\Big)$. Тогда 
	$$f(x) = f(x_{0}) + \sum\limits_{k=0}^{n} \cfrac{b_{k}}{k+1} \cdot (x-x_{0})^{k+1} + o\Big((x-x_{0})^{n+1}\Big), x\to x_{0}$$
\end{theorem}
\begin{proof}
	В силу \hyperlink{thrm5.16}{теоремы о единственности} $b_{k} = \cfrac{(f')^{(k)}(x_{0})}{k!}$.
	
	\hyperlink{thrm5.14}{По формуле Тейлора с остаточным членом в форме Пеано}, примененной к функции $f$, получим, что $f(x) = f(x_{0})+\sum\limits_{k = 1}^{n+1} \cfrac{(f)^{(k)}(x_{0})}{k!} \cdot (x-x_{0})^{k} + o\Big((x-x_{0})^{n+1}\Big), x \to x_{0}$
	
	$$
	j = k-1
	$$
	$$
	f(x) =f(x_{0}) +  \sum\limits_{j = 0}^{n} \cfrac{(f')^{(j)}(x_{0})}{(j+1)!} (x-x_{0})^{j+1} + o\Big((x-x_{0})^{n+1}\Big), x \to x_{0}
	$$
	$$
	f(x) = f(x_{0}) + \sum\limits_{j = 0}^{n} \cfrac{(f')^{(j)}(x_{0})}{j!} \cdot \cfrac{1}{j+1} \cdot (x-x_{0})^{j+1} + o\Big((x-x_{0})^{n+1}\Big), x \to x_{0}
	$$
	$$
	k = j
	$$
	$$f(x) = f(x_{0}) +  \sum\limits_{k = 0}^{n} \cfrac{b_{k}}{k + 1} \cdot (x-x_{0})^{k+1} + o\Big((x-x_{0})^{n+1}\Big), x \to x_{0}.$$
\end{proof}

\begin{problem}
	Пусть $f(x) = f(x_{0}) + a_{1}\cdot x + a_{2}\cdot x^{2} + o(x^{2}), x \to 0.$ Верно ли, что \begin{enumerate}
		\item $f$ непрерывна о точке $0$
		\item $f$ дифференцируема в точке $0$
		\item $f$ дважды дифференцируема в точке $0$
	\end{enumerate}
\end{problem}

\subsection{Разложение основных элементарных функций по формуле Тейлора}
\begin{definition}
	Путем сдвига \hyperlink{def5.13}{формулы Тейлора с центром в точке $x_{0}$} может быть редуцирована к формуле Тейлора с центром в нуле. Тогда такая формула называется \textit{формулой Маклорена.}  То есть если $\exists f^{(n) }(0) \in \R,$ то 
	
	$$f(x) = \sum\limits_{k = 0}^{n} \cfrac{f^{(k)}(0)}{k!} \cdot x^{k} + o(x^{n}), x \to 0$$
\end{definition}

\begin{lemma}
	Пусть $f$ дифференцируема в окрестности 0. Тогда
	\begin{enumerate}
		\item Если $f$ четная, то $f'$ нечетна,
		\item Если $f$ нечетная, то $f'$ четна
	\end{enumerate}
\end{lemma}
\begin{proof}
	Докажем пункт $1.$ так как пункт $2.$ аналогичен.
	
	Пусть $f$~---~ четная функция, тогда.
	
	$$f'(x) = \lim\limits_{\Delta x\to 0} \cfrac{f(x+\Delta x) - f(x)}{\Delta x} = \lim\limits_{\Delta x \to 0} \cfrac{f(-x-\Delta x) - f(-x)}{\Delta x} =\lim\limits_{\Delta x \to 0} \cfrac{f(-x +\Delta x) - f(-x)}{-\Delta x} = -f'(-x)$$
\end{proof}

\begin{note}
	Аналогично можно доказать, что если $f$ четная и дифференцируема в нуле, то ее $f'(0) = 0$
\end{note}

\begin{lemma}
	Пусть $f$~---~ четная и $\exists f^{(2n + 1)}(0) \in \R,$ тогда 
	
	$$f(x) = \sum\limits_{k = 0}^{n} \cfrac{f^{(2k)} \cdot x^{2k}}{(2k)!} + o(x^{2n+1}), x\to 0$$ 
	
	Если $f$~---~ нечетная и $\exists f^{(2n + 2)}(0) \in \R,$ тогда 
	
	$$f(x) = \sum\limits_{k = 0}^{n} \cfrac{f^{(2k+1)} \cdot x^{2k+1}}{(2k+1)!} + o(x^{2n+2}), x\to 0$$ 
\end{lemma}
\begin{proof}
	$f$~---~ четная, но $f'$~---~ нечетная, $f''$~---~ четная, $f'''$~---~ нечетная и так далее. Но в силу предыдущей леммы и замечания $f'(0) = f'''(0) = \ldots = f^{(2n+1)}(0) = 0.$ 
	
	Значит, \hyperlink{thrm5.14}{по формуле Тейлора с остаточным членом в форма Пеано} $$f(x) = \sum\limits_{j = 0}^{2n+1} \cfrac{f^{(j)}}{j!} x^{j} + o(x^{2n+1}), x\to 0$$
	$$= \sum\limits_{k = 0}^{n} \cfrac{f^{(2k)} \cdot x^{2k}}{(2k)!} + o(x^{2n+1}), x\to 0$$
	
	Для нечетной доказательство аналогично
\end{proof}

%\begin{enumerate}
%    \item $\sh x = \sum\limits_{k = 0}^{n} \cfrac{x^{2k+1}}{(2k+1)!} + o(x^{2n+2}), x\to 0$
%    \item[] $\ch x = \sum\limits_{k = 0}^{n} \cfrac{x^{2k}}{(2k)!} + o(x^{2n+1}), x\to 0$
%    \item $\sin x = \sum\limits_{k = 0}^{n} \cfrac{(-1)^{k}}{(2k+1)!} \cdot x^{2k+1} + o(x^{2n+2}), x\to 0$
%    \item[] $\cos x = \sum\limits_{k = 0}^{n} \cfrac{(-1)^{k}}{(2k)!} \cdot x^{2k} + o(x^{2n+1}), x\to 0$
%    \item $(1 + x)^{\alpha} = \sum\limits_{k = 0}^{n} C_{\alpha}^{k}x^{k}  + 0(x^{n}), x\to 0,$
%    \item $ \ln x = \sum\limits_{k = 0}^{n} (-1)^{k} x^{k} + o(x^{n}), x\to 0$
%\end{enumerate}

\begin{enumerate}
	\item $(e^{x}^{(n)})(0) = 1$ 
	$$ \forall n \in \N_{0} \hookrightarrow e^{x} = \sum\limits_{k = 0}^{n} \cfrac{x^{k}}{k!} + o(x^{n}), x\to 0$$
	
	\item $(\sh x)^{(j)}(0) = \begin{cases}
		0, j = 2k \\
		1, j = 2k + 1
	\end{cases}$ 
	
	$$\sh x = \sum\limits_{k = 0}^{n} \cfrac{x^{2k+1}}{(2k+1)!} + o(x^{2n+2}), x\to 0$$
	
	Аналогично $\ch x$
	
	$$\ch x = \sum\limits_{k = 0}^{n} \cfrac{x^{2k}}{(2k)!} + o(x^{2n+1}), x\to 0$$
	
	\item $(\sin x)^{(j)}(0) = \begin{cases}
		0, j = 2k \\
		(-1)^{k}, j = 2k + 1
	\end{cases}$ 
	
	$$\sin x = \sum\limits_{k = 0}^{n} \cfrac{(-1)^{k}}{(2k+1)!} \cdot x^{2k+1} + o(x^{2n+2}), x\to 0$$
	
	Аналогично $\cos x$
	
	$$\cos x = \sum\limits_{k = 0}^{n} \cfrac{(-1)^{k}}{(2k)!} \cdot x^{2k} + o(x^{2n+1}), x\to 0$$
	
	\item $\Big((1 + x)^{\alpha}\Big)^{(n)} = \alpha\cdot\ldots\cdot (\alpha - n+1), \quad \ \alpha \in \R$
	
	$$(1 + x)^{\alpha} = \sum\limits_{k = 0}^{n} C_{\alpha}^{k}x^{k}  + 0(x^{n}), x\to 0,$$
	
	где $\begin{cases}
			C_{\alpha}^{k} = \frac{\alpha\cdot\ldots\cdot (\alpha - n+1)}{k!}, k \in \N \\
			C_{\alpha}^{0} = 1
		\end{cases}$
		
	\item $\ln(1+x)' = \cfrac{1}{1 + x}$ $\quad (1 + x)^{-1} = \sum\limits_{k = 0}^{n} (-1)^{k} x^{k} + o(x^{n}), x\to 0$ 
	
	Тогда $\ln(1 + x)$ получается из \hyperlink{thrm7.4}{теоремы о почленном интегрировании форулы Тейлора} с учетом, что $\ln(1) = 0$
	
	$$\ln(1 + x) = \sum\limits_{k =0}^{n} \cfrac{(-1)^{k} x^{k+1}}{k+1} + o(x^{n}) = \sum\limits_{k=1}^{n+1} \cfrac{(-1)^{k-1} x^{k}}{k} + o(x^{n}) , x\to 0$$
\end{enumerate}
\begin{problem}
	Пусть $x_{0} > 0.$ Разложить $\ln x$ по степеням $(x-x_{0})$ с точностью до $o\Big((x-x_{0})^{n+1}\Big)$
\end{problem}
\begin{solution}
	$\ln (x_{0} + x - x_{0}) = \ln x_{0} + \ln \Big(1 + \cfrac{x-x_{0}}{x_{0}} \Big)$
	
	$t(x) =  \cfrac{x-x_{0}}{x_{0}} , \ \ t(x) \to 0, \ x\to x_{0}$
	
	$ \ln x = \ln x_{0} + \sum\limits_{k = 1}^{n + 1} \cfrac{(-1)^{k} \cdot (x-x_{0})^{k}}{k\cdot x_{0}^{k}} + 0\Big((x-x_{0})^{n+1}\Big), x\to x_{0} $
\end{solution}
\begin{note}
	Заметим, что разложение $\ln (1 + x - 1)$ не является решением данной задачи (если $x_{0} \neq 1$), так как в условии просят разложить по степеням  $x-x_{0}$, а $x-1 \not \to 0б$, при $x-x_{0} \to 0$
\end{note}

\begin{problem}
	Разложить $ \tg x$ в окрестности нуля с точностью до $o(x^{3})$
\end{problem}
\begin{solution}
	$\tg x = \cfrac{\sin x}{\cos x} = \cfrac{x - \cfrac{x^{3}}{6} + o(x^{3})}{1 - \cfrac{x^{2}}{2} + o(x^{3})}$
	
	$t(x) = -\cfrac{x^{2}}{2} + o(x^{3}), t(x) \to 0, x\to 0 $
	
	$(1-t(x))^{-1} = \sum\limits_{k = 0}^{n} (-1)^{k} (t(x))^{k} + o\Big((t(x))^{n}\Big), t(x)\to 0$
	
	Условие  $ t(x) \to 0$ необходимо, так как оно влияет на величину <<поправки >> $o\Big((t(x))^{n}\Big) = \epsilon\Big(t(x)\Big)t^{n}(x),$ где $ \epsilon\Big(t(x)\Big) \to 0, x\to 0$
	
	$\Big(1 - \cfrac{x^{2}}{2} + o(x^{3})\Big)^{-1} = (1-t(x))^{-1} = 1 + \cfrac{x^{2}}{2} + o(x^{3}), x\to 0 $
	
	$$\tg x = \Big(x - \cfrac{x^{3}}{6} + o(x^{3})\Big)\Big(1 + \cfrac{x^{2}}{2} + o(x^{3})\Big) = x-\cfrac{x^{3}}{6} + \cfrac{x^{3}}{2} + o(x^{3}) = x + \cfrac{x^{3}}{3} +  o(x^{3}), x\to 0 $$
\end{solution}
