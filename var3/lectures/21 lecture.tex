\begin{theorem}
	\hypertarget{thrm5.25}{(Необходимое условие экстремума в терминах второй производной)} Пусть $\exists f''(x_{0})$. Тогда если $x_{0}$~---~ точка локального экстремума, то $f'(x_{0}) = 0$ и, если $x_{0}$ ~---~ локальный максимум, $f''(x_{0}) \leq 0$, а если локальный минимум, то $f''(x_{0}) \geq 0$ .
\end{theorem}
\begin{proof}
	То, что $f'(x_{0}) = 0$, следует из \hyperlink{thrm5.9}{теоремы Ферма}. 
	
	Докажем утверждение для локального минимума. Предположим противное: пусть если $x_{0}$~---~ локальный минимум, то $f''(x_{0}) < 0$. Но из этого условия по предыдущей теореме при производной порядка $n = 2$ получаем, что $x_{0}$~---~ строгий локальный максимум. Получили противоречие. Значит, наше предположение неверно и если $x_{0}$~---~ локальный минимум, то $f''(x_{0}) \geq 0$.
	
	Доказательство для локального максимума аналогично.  
\end{proof}

\paragraph{Выпуклости и точки перегиба}
.

\sidefig(11 cm)(6 cm)	
{\begin{flushleft}
	\normalsize{
		\begin{definition}
			Функция $f: (a, b) \mapsto \R$ называется \textit{нестрого выпуклой вниз (вверх) на $(a, b)$}, если $\forall x_{1}, x_{2} \in (a, b) \hookrightarrow f(t\cdot x_{1} + (1-t)\cdot x_{2}) \underset{\geq}{(\leq)} t\cdot f(x_{1}) + (1- t)\cdot f(x_{2}) \quad \forall t\in [0, 1]$
		\end{definition}
		\begin{definition}
				Функция $f: (a, b) \mapsto \R$ называется \textit{строго выпуклой вниз (вверх) на $(a, b)$}, если $\ \forall x_{1}, x_{2} \in (a, b) \hookrightarrow f(t\cdot x_{1} + (1-t)\cdot x_{2}) \underset{(>)}{(<)} t\cdot f(x_{1}) + (1- t)\cdot f(x_{2}) \quad \forall t\in (0, 1)$
		\end{definition}}
\end{flushleft}}
{
	\begin{tikzpicture}
	% Рисуем сетку
	\draw[help lines, step=0.25, dotted]
	(0,0) grid (4.5,4);
	% Начало координат
	\draw[->, thin] (0,0) -- (4.55,0)
	node[below] {$x$}; % Ox
	\draw[->, thin] (0,0) -- (0, 4.05)
	node[left] {$y$}; % Oy
	
	\draw[line width =.05cm](0.25, 3.5) parabola bend (2.5, 1.5)(4, 2.5);
	
	\draw[line width =.03cm, blue] (1, 2.4) -- (3.5, 1.95);
	
	\draw[dotted, line width =.03cm] (2.5, 0) -- (2.5, 2.2);
	\draw[dotted, line width =.03cm] (0.25, 0) -- (0.25, 3.5);
	\draw[dotted, line width =.03cm] (1, 0) -- (1, 2.4);
	\draw[dotted, line width =.03cm] (3.5, 0) -- (3.5, 2);
	\draw[dotted, line width =.03cm] (4, 0) -- (4, 2.5);
	\draw[dotted, line width =.03cm] (1, 2.4) -- (0, 2.4);
	\draw[dotted, line width =.03cm] (0, 1.9) -- (3.5, 1.9);
	
	\draw[fill=blue, blue] (3.5, 1.95) circle(1.5 pt);
	\draw[fill=blue, blue] (1, 2.4) circle(1.5 pt);
	\draw[fill=blue, blue] (2.5, 2.12) circle(1.5 pt);	
	
	\node[below] at (0.25, 0) {$a$};
	\node[below] at (1, 0) {\small$x_{1}$};
	\node[below] at (3.5, 0) {$x_{2}$\normalsize};
	\node[below] at (2.2, 0) {\tiny $x_{2} + t\cdot(x_{2}-x_{1})$\normalsize};
	\node[below] at (4, 0) {$b$};
	\node[left] at (0, 2.4) {\tiny$f(x_{1})$\normalsize};
	\node[left] at (0, 1.9) {\tiny$f(x_{2})$\normalsize};
	
	\node at (2.7, 3.5) {\tiny$\begin{gathered}
			x(t) = x_{2} + t \cdot (x_{1} - x_{2})\hfill \\ 
			y(t) = f(x_{2}) + t\cdot (f(x_{1}) - f(x_{2}))
			\end{gathered}$\normalsize};
	
	\draw[->, line width=.03cm] (3, 2.1) arc(0:9.3:6);
	
	\draw[thin] (0.25, -0.1) -- (0.25, 0.1);
	\draw[thin] (4, -0.1) -- (4, 0.1);
\end{tikzpicture}
}

\begin{note}
	Выпуклые вверх функции иногда называют \textit{вогнутыми}
\end{note}

\begin{problem}
	Доказать, что если $f$ выпукла вверх (вниз) на $(a, b),$ то она непрерывна на $(a, b)$ 
\end{problem}

\begin{theorem}
	Пусть $f:(a, b) \mapsto \R$ дважды дифференцируема на $(a,b)$. Тогда
	\begin{enumerate}
		\item $f$ выпукла вниз на $(a, b) \Leftrightarrow f''(x) \geq 0 \ \forall x\in (a, b)$
		\item $f$ выпукла вверх на $(a, b) \Leftrightarrow f''(x) \leq 0 \ \forall x\in (a, b)$
	\end{enumerate}
\end{theorem}
\begin{proof}
	Докажем пункт $1.$, так как пункт $2.$ аналогичен.
	
	$\underline{\textrm{Шаг 1.}}$ Пусть $f$ выпукла вниз. Фиксируем $x_{0} \in (a, b)$. Введем $u \in (0,\min\{x_{0} - a, b-x_{0}\}),$
	
	$$\begin{gathered}
		x_{1} = x_{0} - u \\
		x_{2} = x_{0} - u
	\end{gathered} \Rightarrow x_{0} = \dfrac{x_{1} + x_{2}}{2}$$ 
	
	Разложим функцию $f$ \hyperlink{thrm5.14}{по формуле Тейлора с остаточным членом в форме Пеано}:
	
	$$f(x_{1}) = f(x_{0}) + f'(x_{0})(-u) + \dfrac{f''(x_{0})\cdot u^{2}}{2} + o(u^{2}), u\to0$$
	$$f(x_{2}) = f(x_{0}) + f'(x_{0})(u) + \dfrac{f''(x_{0})\cdot u^{2}}{2} + o(u^{2}), u\to0$$
	
	Используем условие выпуклости при $t = \dfrac{1}{2}$:
	$$f(x_{0}) \leq \dfrac{f(x_{1}) + f(x_{2})}{2} = \dfrac{2f(x_{0}) + f''(x_{0}) + o(u^{2})}{2} = f(x_{0}) + f''(x_{0})\dfrac{u^{2}}{2} + o(u^{2})$$
	$$f(x_{0}) \leq f(x_{0}) + f''(x_{0})\dfrac{u^{2}}{2} + o(u^{2}) \Rightarrow f''(x_{0})\dfrac{u^{2}}{2} + o(u^{2}) \geq 0 \Rightarrow \dfrac{f''(x_{0})}{2} + 0(1)$$
	
	\hyperlink{thrm4.8}{Переходя к пределу в неравенстве} получим $f''(x_{0}) \geq 0$
	
	$\underline{\textrm{Шаг 2.}}$ Пусть наоборот $f''(x) \geq 0 \ \forall x\in (a, b)$. Покажем, что $f$ выпукла вниз на $(a, b).$
	
	Фиксируем произвольные точки $x_{1}, x_{2} \in (a, b)$ и $t \in (0, 1).$ Покажем, что выполняется $$f(\underbrace{t\cdot x_{1} (1-t) \cdot x_{2}}_{x_{0}}) \leq t\cdot f(x_{1}) + (1-t) \cdot f(x_{2})$$
	
	Воспользуемся \hyperlink{thrm5.15}{формулой Тейлора с остаточным членом в форме Лагранжа}:
	
	$$f(x_{1}) = f(x_{0}) + f'(x_{0})(x_{1}-x_{0}) + \dfrac{f''(\xi_{1})}{2!} (x_{1}-x_{0})^{2}, \ \xi_{1} \in (x_{1}, x_{0}) $$
	$$f(x_{2}) = f(x_{0}) + f'(x_{0})(x_{2}-x_{0}) + \dfrac{f''(\xi_{2})}{2!} (x_{2}-x_{0})^{2}, \ \xi_{2} \in (x_{0}, x_{2}) $$
	
	Так как $\dfrac{f''(\xi_{1})}{2!}  \geq 0$ и $\dfrac{f''(\xi_{2})}{2!} \geq 0,$ то $\begin{gathered}
		f(x_{1}) \geq f(x_{0}) + f'(x_{0}) (x_{1}-x_{0}) \\
		f(x_{2}) \geq f(x_{0}) + f'(x_{0}) (x_{2}-x_{0}). 
	\end{gathered}$  Следовательно,
	$$t\cdot f(x_{1}) + (1-t) \cdot f(x_{2}) \geq t \cdot f(x_{0}) +(1-t) \cdot f(x_{0}) + t\cdot f'(x_{0})(x_{1}-x_{0}) + (1-t) \cdot f'(x_{0})(x_{2} - x_{0}) $$
	$$t\cdot f(x_{1}) + (1-t) \cdot f(x_{2}) \geq f(x_{0}) + f'(x_{0}) \underbrace{\Big(t\cdot (x_{1} - x_{0}) + (1-t)\cdot (x_{2}-x_{0})\Big)}_{0} = f(x_{0})$$
	
	Так как $t \in (0, 1)$ и точки $x_{1}, x_{2}$  выбраны произвольно, то теорема доказана.
\end{proof}

\sidefig(10 cm)(6 cm)	
{\begin{flushleft}
\normalsize{		
\begin{definition}
	 Пусть $f\in C\Big((a, b)\Big)$ и $\exists f'(x_{0}) \in \overline{\R}.$ Пусть выполняется одно их двух условий: \begin{enumerate}
		\item $f$ выпукла вниз на $U_{\delta}^{-}(x_{0})$ и выпукла вверх на $U_{\delta}^{+}(x_{0})$
		\item $f$ выпукла вверх на $U_{\delta}^{-}(x_{0})$ и выпукла вниз на  $U_{\delta}^{+}(x_{0})$
	\end{enumerate}
	
	Тогда $x_{0}$ называется \textit{точкой перегиба графика функции $f$}
\end{definition}}
\end{flushleft}}
{
\begin{tikzpicture}[scale = 1.1]
	% Рисуем сетку
	\draw[help lines, step=0.25, dotted]
	(-2,-2) grid (2,2);
	% Начало координат
	\draw[->, thin] (-2.05,0) -- (2.05,0)
	node[below] {$x$}; % Ox
	\draw[->, thin] (0,-2.05) -- (0,2.05)
	node[left] {$y$}; % Oy
	
	\draw[blue, line width=.03cm] (0.0) parabola bend (2, 1.4) |-(0,0);
	\draw[blue, line width=.03cm] (0.0) parabola bend (-2, -1.4) |-(0,0);
	
	\draw[->] (0.5, -0.5) arc(40:60:1);
	
	\node at (1,-0.7) {\tiny Точка перегиба}
	
	\draw[fill = blue] (0, 0) circle (1.5 pt);
	\node[above left] at (0,0) {$O$};
	\node[below] at (0, -2) {$sign x\cdot \sqrt{|x|}$};
\end{tikzpicture}
}

\begin{theorem}
	\hypertarget{thrm5.27}{(Критерий точки перегиба)} Пусть $f \in C\Big(U_{\delta}(x_{0})\Big)$ и $\exists f'(x_{0}) \in \overline{\R}$, $f$ дважды дифференцируема в $\mathring{U}_{\delta}(x_{0}).$ Тогда $x_{0}$~---~ точка перегиба графика функции $f\Leftrightarrow$ $$\Leftrightarrow \left[\begin{gathered}
	f''(x) \geq 0\ \forall x\in (x_{0} - \delta, x_{0}) \ \textrm{и} \ f''(x) \leq 0 \ \forall (x_{0}, x_{0} + \delta) \\
	f''(x) \leq 0\ \forall x\in (x_{0} - \delta, x_{0}) \ \textrm{и} \ f''(x) \geq 0 \ \forall (x_{0}, x_{0} + \delta) 
	\end{gathered}\right.$$
\end{theorem}
\begin{proof}
	Доказательство состоит в применении определения и критерия выпуклости.
\end{proof}

\sidefig(10 cm)(6 cm)	
{\begin{flushleft}
\normalsize
\begin{definition}
	Пусть $f$ дифференцируема в точке $x_{0}$. Будем говорить что, график функции $f$ \textit{переходит с одной стороны касательной на другую при переходе через точку $x_{0}$}, если \footnotesize $\exists \delta > 0: \left[\begin{gathered}
	y_{\textrm{кас.}}(x) \leq f(x) \ \forall x\in (x_{0} - \delta, x_{0}) \ \textrm{и} \ y_{\textrm{кас.}}(x) \geq f(x) \ \forall x\in(x_{0}, x_{0} + \delta) \\
	y_{\textrm{кас.}}(x) \geq f(x) \ \forall x\in (x_{0} - \delta, x_{0}) \ \textrm{и} \ y_{\textrm{кас.}}(x) \leq f(x) \ \forall x\in(x_{0}, x_{0} + \delta)
	\end{gathered}\right.$
\end{definition}
\end{flushleft}}
{
\begin{tikzpicture}[scale = 1]
	% Рисуем сетку
	\draw[help lines, step=0.25, dotted]
	(-2,-2) grid (2,2);
	% Начало координат
	\draw[->, thin] (-2.05,-2) -- (2.05,-2)
	node[below] {$x$}; % Ox
	\draw[->, thin] (-2,-2.05) -- (-2,2.05)
	node[left] {$y$}; % Oy
	
	\draw[blue, line width=.03cm] (0.0) parabola bend (2, 1.4) |-(0,0);
	\draw[blue, line width=.03cm] (0.0) parabola bend (-2, -1.4) |-(0,0);
	
	\draw[line width =.03cm] (-1.5,-2) -- (1.5,2);
	\node[above, left] at (1, 1.5) {$y_{\textrm{кас.}(x)}$};
	
	\draw[fill = blue] (0, 0) circle (1.5 pt);
	\draw[dotted, line width=.03cm] (0,0) -- (0, -2);
	\node[below] at (0,-2) {$x_{0}$};
\end{tikzpicture}
}

\begin{theorem}
	Пусть $f$ дважды дифференцируема в $\mathring{U}_{\delta}(x_{0})$ и дифференцируема в точке $x_{0}$. Тогда если $x_{0}$~---~ точка перегиба графика функции $f,$ то график переходит с одной стороны касательной на другую. 
	
	Обратное неверно.
\end{theorem}
\begin{proof}
	Пусть $x_{0}$~---~ точка перегиба графика. Тогда в силу \hyperlink{thrm5.27}{критерия точки перегиба} $\exists \delta > 0: f''(x) \leq 0 \ \forall (x_{0} - \delta, x_{0})$ и $f''(x) \geq 0 \ \forall x\in (x_{0, x_{0} + \delta})$ (второй случай рассматривается аналогично).
	
	\hyperlink{thrm5.15}{По формуле Тейлора с остаточным членом в форме Лагранжа} имеем:
	$$\forall x\in (x_{0} - \delta, x_{0}) \hookrightarrow f(x) = \underbrace{f(x_{0}) + f'(x_{0})(x-x_{0})}_{y_{\textrm{кас.}}} + \dfrac{f''(\xi)(x-x_{0})^{2}}{2}, \ \dfrac{f''(\xi)(x-x_{0})^{2}}{2} \leq 0 \Rightarrow$$
	
	$$ \Rightarrow f(x) \leq y_{\textrm{кас.}}(x) \ \forall x\in (x_{0}-\delta, x_{0})$$
	
	$$\forall x\in (x_{0}, x_{0} + \delta) \hookrightarrow f(x) = \underbrace{f(x_{0}) + f'(x_{0})(x-x_{0})}_{y_{\textrm{кас.}}} + \dfrac{f''(\xi)(x-x_{0})^{2}}{2}, \ \dfrac{f''(\xi)(x-x_{0})^{2}}{2} \geq 0 \Rightarrow$$
	
	$$ \Rightarrow f(x) \geq y_{\textrm{кас.}}(x) \ \forall x\in (x_{0}, x_{0} + \delta)$$
	
	Следовательно, график перешел с одной стороны касательной на другую.
	
	\textit{Почему обратное неверно?}

	\sidefig(7.5 cm)(8.5 cm)	
	{\begin{flushleft}
			\normalsize{
	\begin{example}
		$f(x) = \begin{cases}
			x^{3}\Big(2+\sin\dfrac{1}{x}\Big), \ x\neq 0, \\
			0, \ x=0
		\end{cases}$
		
		$$f'(0) = 0$$
		
		Горизонтальная касательная в точке $0$: $y_{\textrm{кас.}}  = 0$
			
		$$\begin{cases}
			f(x) > 0, \ x> 0 \\
			f(x) < 0, x< 0
		\end{cases}$$
		
		Но точка $0$ не является точкой перегиба графика $f$, так как $f$ не обладает какой-либо выпуклостью в левой или правой полуокрестности точки $0$.
	\end{example}}
	\end{flushleft}
	}
	{
		\begin{tikzpicture}[>=stealth]
		\begin{axis}[
			restrict y to domain = -0.4:0.4,
			restrict x to domain = -0.5:0.5,
			samples=2000, 
			axis lines*=middle, 
			xtick={0.5}, 
			ytick={0.4}, 
			axis line style={->},
			xticklabels={$5 \cdot 10^{-2}$},
			yticklabels={$4 \cdot 10^{-4}$},
			]
			\addplot[dotted] {((x/10)^(3))*1000};
			\addplot[blue, dotted] {((x/10)^(3))*3000};
			\addplot[color = red]{(((x/10)^(3))*(2+sin(deg(10/x)))*1000};
		\end{axis}
		\draw [color = red, restrict y to domain = -0.0001:0.0001, restrict x to domain = -0.04:0.04, samples=500] plot (\x, {(\x^(3))*(2+sin((1/\x)r)});
	\end{tikzpicture}
	}
\end{proof}

\begin{center}
	\textbf{Асимптоты}
	
	    \centering
	\begin{tabular}{cс}
	\begin{tikzpicture}[scale=0.25]
		% Рисуем сетку
		\draw[help lines, step=1.25, dotted]
		(-10,-10) grid (10,10);
		% Начало координат
		\draw[->, thin] (-10.05,0) -- (10.05,0)
		node[below] {$x$}; % Ox
		\draw[->, thin] (-1.25,-10.05) -- (-1.25,10.05)
		node[left] {$y$}; % Oy
		
		\draw[blue, line width=.03cm] plot[domain=0.61:10, samples = 200]  (\x, {1/(\x - 0.5) + 0.5});
		\draw[blue, line width=.03cm] plot[domain=-10:-0.61, samples = 200]  (\x, {1/(\x + 0.5) - 0.5});
		
		\draw[line width=.03cm, dotted] (0, -10) -- (0, 10);
		
		
		\node[above left] at (-1.25,0) {$O$};
		\node[below] at  (0,0) {$x_{0}$};
		\node[below] at (0, -10.5) {\textrm{Вертикальная асимптота}}
	\end{tikzpicture}
		&
	\begin{tikzpicture}[scale=0.25]
		% Рисуем сетку
		\draw[help lines, step=1.25, dotted]
		(-10,-10) grid (10,10);
		% Начало координат
		\draw[->, thin] (-10.05,0) -- (10.05,0)
		node[below] {$x$}; % Ox
		\draw[->, thin] (-1.25,-10.05) -- (-1.25,10.05)
		node[left] {$y$}; % Oy
		
		\draw[blue, line width=.03cm] plot[domain=2.55:10, samples = 200]  (\x, {((\x)^2+1)/(\x - 2)-5});
		\draw[blue, line width=.03cm] plot[domain=-6:1.4, samples = 200]  (\x, {((\x)^2+1)/(\x- 2)-5});
		
		\draw[line width=.03cm, dotted] (-7, -10) -- (10, 7);
		
		
		\node[above left] at (-1.25,0) {$O$};
		\node[below] at  (0,0) {$x_{0}$};
		\node[below] at (0, -10.5) {\textrm{Наклонная асимптота}}
	\end{tikzpicture}
	\end{tabular}
\end{center}

\begin{definition}
	Будем говорить, что график функции $f$ имеет \textit{вертикальную асимптоту в точке $x_{0}$}, если $\begin{cases}
		\lim\limits_{x\to x_{0}+0} f(x)= \pm \infty \\
		\lim\limits_{x\to x_{0}-0} f(x)= \pm \infty
	\end{cases}$
\end{definition}
\begin{definition}
	Прямая $y = k \cdot x + b, \ b, k \in \R$ называется \textit{наклонной асимптотой графика функции} $f$: $(A, \infty)$, если $\exists \lim\limits_{x\to +\infty}(f(x) - kx - b) = 0$.
	
	Аналогично определяется \textit{наклонная асимптота на $-\infty$} 
\end{definition}}

\begin{theorem}
	\hypertarget{thrm5.29}{(Критерий асимптоты)} 
	
	Прямая $y = kx + b$~---~ асимптота функции $f$: $(A, +\infty) \ \Leftrightarrow (*)\ \begin{cases}
		\exists \lim\limits_{x \to +\infty} \dfrac{f(x)}{x} = k \\
		\exists \lim\limits_{x\to +\infty} (f(x) - kx) = b
		\end{cases}$
\end{theorem}
\begin{proof}
	$\underline{\textrm{Шаг 1.}}$ Пусть $y = kx + b$~---~ асимптота. Следовательно, 
	
	$$\exists \lim\limits_{x\to +\infty} (f(x) - kx) -b = 0 \Rightarrow \exists \lim\limits_{x\to +\infty} (f(x) - kx) = b \Rightarrow f(x) -kx = b + o(1), x\to +\infty$$
	
	Разделив все на $x,$ получаем 
	$$\dfrac{f(x)}{x} = k + \dfrac{b}{x} + \dfrac{1}{x}o(1), x\to \infty.$$ 
	
	Следовательно, $\lim\limits_{x \to +\infty} \dfrac{f(x)}{x} = k$.
	
	 $\underline{\textrm{Шаг 2.}}$ Пусть обратно выполнена $(*)$. Тогда $\lim\limits_{x\to +\infty} (f(x) - kx) = b.$
\end{proof}

\paragraph{План построения графика}
\begin{enumerate}
	\item Найти область определения.
	\item Найти точки пересечения с осями координат.
	\item Построить асимптоты, если они есть.
	\item Нарисовать эскиз графика.
	\item Найти $f'(x),$ точки экстремума, интервалы возрастания и убывания функции.
	\item Найти $f''(x)$, интервалы выгнутости и вогнутости, точки перегиба.
	\item Строим уточненный график.
\end{enumerate}


\section{Первообразная, неопределенный интеграл, полиномы, комплексные числа}
\begin{definition}
	Будем говорить, что $f$: $(a, b) \mapsto \R$ имеет \textit{первообразную} $F$ на $(a, b)$, если $F$~---~дифференцируема на $(a, b)$ и $\forall x \in (a, b)$ $F'(x) = f (x)$.
\end{definition}
\begin{note}
	Бессмысленно считать первообазную в точке, мы всегда рассматриваем интервалы.
\end{note}
\begin{note}
	Если $f$: $(a, b) \mapsto \R$ имеет устранимый разрыв или разрыв первого рода, то она не имеет на $(a, b)$ первообразной (следствие из теоремы Лагранжа о среднем).
\end{note}
\begin{definition}
	\textit{Неопределенным интегралом} функции $f$ на $(a, b)$ будем называть множество всех первообразных на $(a, b)$ и записывать $\displaystyle \int f(x) \,dx$.
\end{definition}
\begin{lemma}
	Пусть $f$ дифференцируема на $(a, b)$. Тогда $f' (x) = 0 \  \forall x \in (a, b) \Leftrightarrow f (x) \equiv const$.
\end{lemma}
\begin{proof}
	1. Eсли $f (x) = t \ \forall x \in (a, b),$ то $f'(x) = 0 \ \forall x \in (a, b).$
	
	2. Пусть $f'(x) = 0 \ \forall x \in (a, b).$ Рассмотрим произвольные $x_{1}, x_{2} \in (a, b)$, тогда по теореме Лагранжа $\exists \xi \in (a, b)$: $f (x_{1}) - f (x_{2}) = f' (\xi) (x_{1} - x_{2}) = 0 \Rightarrow f (x_{1}) = f (x_{2}).$ А так как $x_{1}$ и $x_{2}$ были выбраны из $(a, b)$ произвольно, то $f (x) = const$ на $(a, b)$.
\end{proof}
\begin{corollary}
	\hypertarget{coll6.1}{Пусть $F$~---~первообразная $f$ на $(a, b)$. Тогда $\displaystyle \int f (x) \, dx = \{ F (x) + C: C \in \R \}$}.
\end{corollary}
\begin{proof}
	Пусть $F_{1}$ и $F_{2}$~---~две различные первообразные $f$. Тогда $F'_{1}(x) - F'_{2}(x) = f (x) - f (x) = 0 \Rightarrow$ по лемме $2.1$ $F_{1} - F_{2} = const \Rightarrow$ первообразные отличаются только на $C \in \R$.
\end{proof}