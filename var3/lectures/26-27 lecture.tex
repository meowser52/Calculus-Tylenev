\begin{note}
    $\overline{B}_{r} (x)$ не всегда совпадает с замыканием открытого шара. Также стоит отметить, что происходит такая коллизия обозначений.
\end{note}
\begin{example}
    Пусть $X = [0; 1] \cap \{ 2\}, \rho(x, y) = |x - y|$. Возьмём $B_{1} (1) = (0; 1]$. $\text{cl}B_{1} (1) = [0; 1]$, но $\overline{B}_{1} (1) = X$, то есть замкнутый шар может быть <<шире>>, чем замыкание открытого шара.
\end{example}
\begin{lemma}
    \hypertarget{lemm7.2}{Пусть $(X, \rho)$~---~метрическое пространство. Тогда $\forall \E \subset X \hookrightarrow$
    \begin{enumerate}
        \item $X \backslash \text{cl}\E = \text{int}(X \backslash \E)$;
        \item $X \backslash \text{int}\E = \text{cl}(X \backslash \E)$.
    \end{enumerate}}
\end{lemma}
\begin{proof}
    Покажем (2), так как (1) аналогично.
    
    Пусть $x^{*} \in (X \backslash \text{int}\E)$. Тогда 
    $\begin{cases}
        x^{*} \in X \\
        !(x^{*} \in \text{int}\E)
    \end{cases}
    \Leftrightarrow \begin{cases}
        x^{*} \in X \\
        \forall \epsilon > 0 \ B_{\epsilon} (x^{*}) \not\subset \E
    \end{cases}
    \Leftrightarrow$ 
    
    $$\Leftrightarrow \forall \epsilon > 0 \ B_{\epsilon} (x^{*}) \cap (X \backslash \E) \neq \varnothing \Leftrightarrow x^{*}\text{~---~точка прикосновения }X\backslash\E \Leftrightarrow x^{*} \in \text{cl}(X \backslash \E).$$
\end{proof}
\begin{corollary}
    Пусть $(X, \rho)$~---~метрическое пространство. Тогда множество $\E$ замкнуто $\Leftrightarrow (X\backslash\E)$ открыто.
\end{corollary}
\begin{proof}
    Так как $\E$ замкнуто, то оно совпадает со своим замыканием. В силу \hyperlink{lemm7.2}{предыдущей леммы} $X \backslash \E = X \backslash \text{cl}\E = \text{int}(X\backslash\E)$~---~$X\backslash\E$ открыто.
\end{proof}
\begin{definition}
    Пусть $\E$~---~множество в метрическом пространстве. Тогда \textit{границей множества} назовём $\text{cl}\E \backslash \text{int}\E$ и будем обозначать $\partial \E$.
\end{definition}
\begin{lemma}
    Пусть $(X, \rho)$~---~метрическое пространство, $\E \subset X$. Тогда $x_{0} \in \partial \E \Leftrightarrow \newline \Leftrightarrow \forall \epsilon > 0 \hookrightarrow
    \begin{cases}
        B_{\epsilon} (x_{0}) \cap \E \neq \varnothing \\
        B_{\epsilon} (x_{0}) \cap (X \backslash \E) \neq \varnothing
    \end{cases}$
\end{lemma}
\begin{proof}
    Так как $x_{0} \in \partial \E$, то по определению $\begin{cases}
        \forall \epsilon > 0 \ B_{\epsilon} (x_0) \cap \E \neq \varnothing \\
        !(\exists \epsilon > 0: B_{\epsilon} (x_{0}) \subset \E)
    \end{cases} \Leftrightarrow \newline \Leftrightarrow \begin{cases}
        \forall \epsilon > 0 \ B_{\epsilon} (x_0) \cap \E \neq \varnothing \\
        \forall \epsilon > 0 \ B_{\epsilon} (x_0) \cap (X \backslash \E) \neq \varnothing
    \end{cases}$, что нам и нужно было.
\end{proof}
\begin{theorem}
    (Критерий точки прикосновения) $x_{0}$~---~точка прикосновения множества $\E \Leftrightarrow \exists \{ x_{n}\} \subset \E$: $\rho(x_{n}, x_{0})\to 0$, $n\to \infty$.
\end{theorem}
\begin{proof}
    Ровно \hyperlink{thm3.1}{такое же}, как и на числовой прямой.
\end{proof}
\begin{theorem}
    (Критерий предельной точки) $x_{0}$~---~предельная точка множества $\E \Leftrightarrow 
    \begin{cases}
        x_{n} \neq x_{0} \ \forall n \in \N \\
        \rho(x_{n}, x_{0})\to 0, n\to \infty
    \end{cases}$
\end{theorem}
\begin{proof}
    Точно такое же, как и на числовой прямой.

    \textbf{Я не смог его найти.}
\end{proof}
\begin{problem}
    Пусть $(X, \rho)$~---~метрическое пространство. Доказать, что $\forall \E \subset X \hookrightarrow \E \cup \partial \E = \text{cl}\E$, $\E \backslash \partial \E = \text{int}\E$.
\end{problem}
\begin{problem}
    Верны ли следующие включения?
    \begin{enumerate}
        \item $\text{int}(\E_{1} \cup \E_{2}) \subset (\text{int}\E_{1} \cup \text{int}E_{2})$;
        \item $(\text{int}\E_{1} \cup \text{int}E_{2}) \subset \text{int}(\E_{1} \cup \E_{2})$;
        \item $\text{cl}(\E_{1} \cup \E_{2}) \subset (\text{cl}\E_{1} \cup \text{cl}E_{2})$;
        \item $(\text{cl}\E_{1} \cup \text{cl}E_{2}) \subset \text{cl}(\E_{1} \cup \E_{2})$;
        \item $\partial(\E_{1} \cup \E_{2}) \subset (\partial\E_{1} \cup \partial\E_{2})$;
        \item $(\partial\E_{1} \cup \partial\E_{2}) \subset \partial(\E_{1} \cup \E_{2})$.
    \end{enumerate}
\end{problem}
\begin{example}
    Может быть так, что $\partial \E = \R$, а $\text{int}\E = \varnothing$. К примеру, $\E = \Q$.
\end{example}
\begin{definition}
    Пусть $(X, \rho)$~---~метрическое пространство, $\E \subset X$. Множество $\E$ называется \textit{ограниченным}, если $\exists B_{R} (x_{0})$: $\E \subset B_{R} (x_{0})$.
\end{definition}
\begin{definition}
    Пусть $(X, \rho)$~---~метрическое пространство, $\E \subset X$. Множество $\E$ \textit{вполне ограничено}, если
    $$ \forall \epsilon > 0 \ \exists \text{конечное число точек } \{ x_{1}, \ldots, x_{N (\epsilon)}\}: \E \subset \bigcup_{i = 1}^{N (\epsilon)} B_{\epsilon} (x_{i}).$$
\end{definition}
\begin{note}
    И эти точки называются $\epsilon$-сетью для $\E$.
\end{note}
\begin{lemma}
    Если множество $\E$~---~вполне ограничено, то $\E$ ограничено.
\end{lemma}
\begin{proof}
    Так как $\E$ вполне ограничено, то $\forall \epsilon > 0 \ \exists$ конечная $\epsilon$-сеть. Значит и для $\epsilon = 1 \ \exists$ конечная $1$-сеть $\{ x_{1}, \ldots, x_{N} \}$. Мы можем взять, к примеру, $M = N + \max\limits_{i \in \{1, \ldots, N\}} \rho(x_{1}, x_{i})$, тогда $\E \subset B_{M} (x_{1})$ по неравенству треугольника $\Rightarrow \E$ ограничено.
\end{proof}
\begin{lemma}
    Пусть $(X, \rho)$~---~метрическое пространство, а $\{ x_{n} \} \subset X$ сходится к $x^{*} \in X$. Тогда $\{ x_{n} \}$ фундаментальна.
\end{lemma}
\begin{proof}
    Пусть $\{ x_{n} \} \to x^{*}$, $n \to \infty$, запишем это:
    $$ \forall \epsilon > 0 \ \ \exists N (\epsilon) \in \N: \forall n \geq N (\epsilon) \hookrightarrow \rho(x_{n}, x^{*}) < \epsilon.$$
    Тогда $\forall \epsilon > 0$ $\exists \Tilde{N} (\epsilon) = N (\frac{\epsilon}{2})$: $\forall n, m \geq \Tilde{N} (\epsilon) \hookrightarrow \rho(x_n, x_m) \leq \rho(x_n, x^*) + \rho(x^*, x_m) < \frac{\epsilon}{2} + \frac{\epsilon}{2} = \epsilon$.
\end{proof}
\begin{lemma}
    \hypertarget{lemm7.6}{Пусть $(X, \rho)$~---~метрическое пространство, $K \subset X$. Если множество $K$~---~компакт, то $K$~---~вполне ограниченное множество.}
\end{lemma}
\begin{proof}
    Предположим противное. Будем считать, что $K \neq \varnothing$, так как пустое множество по определению компакт. Запишем отрицание к определению вполне ограниченного множества:
    $$ \exists \epsilon > 0: \forall \text{ конечное набора точек } \{ x_{1}, \ldots, x_{N (\epsilon)}\}: \E \not\subset \bigcup_{i = 1}^{N (\epsilon)} B_{\epsilon} (x_{i}).$$

    Пусть $x_{1}\in K$. Возьмём $B_{\epsilon} (x_{1})$, он не покрывает $K \Rightarrow \exists x_{2}$: $x_{2} \in (K \backslash B_{\epsilon} (x_{1}))$, но $\{ x_{1}, x_{2} \}$ тоже не $\epsilon$-сеть $\Rightarrow \exists x_{3} \in \left(K \backslash (B_{\epsilon} (x_{1}) \cup B_{\epsilon} (x_{2}))\right)$. Так продолжим по индукции. Пусть таким образом мы построили точки $x_{1}, \ldots, x_{n}$: $K \backslash \bigcup\limits_{i = 1}^{n} B_{\epsilon} (x_{i}) \neq \varnothing$, тогда возьмём $x_{n + 1} \in \left( K \backslash \bigcup\limits_{i = 1}^{n} B_{\epsilon} (x_{i})\right)$, а $\{ x_{1}, \ldots, x_{n + 1}\}$ снова не $\epsilon$-сеть. То есть мы построили последовательность $\{ x_{n} \} \subset K$: $\rho(x_{i}, x_{j}) \geq \epsilon \ \forall i \neq j \Rightarrow$ любая её подпоследовательность не является фундаментальной, а значит сходящейся, а значит $K$~---~не компакт. Противоречие.
\end{proof}
\begin{definition}
    Метрическое пространство $(X, \rho)$ называется \textit{полным}, если любая его фундаментальная последовательность сходится к некоторой точке этого пространства, в противном случае пространство называется \textit{неполным}.
\end{definition}
\begin{examples}$\ $

    $X = \R$, $\rho(x, y) = |x - y|$~---~полное (критерий Коши мы доказывали ранее).

    $X = \Q$, $\rho(x, y) = |x - y|$~---~неполное.
\end{examples}
\begin{proposition}
    В любом метрическом пространстве любая сфера~---~замкнутое множество.
\end{proposition}
\begin{example}
    Множество, которое ограничено и замкнуто, но не является компактом:

    $X = \{ x = (x_{1}, \ldots, x_{n}, \ldots)$: $\exists \lim\limits_{n\to \infty} x_{n} = 0\}$, $\rho(x, y) = \sup\limits_{i \in \N}|x_{i} - y_{i}|$.

    Рвссмотрим $S_{1} (0) := \{ x \in X$: $\sup\limits_{i \in \N} |x_{i}| = 1\}$, то есть единичную сферу. Оно является ограниченным и замкнутым множеством, но не вполне ограниченным, а значит, не компактом.

    \underline{Замечание.} Для доказательства того, что множество не является вполне ограниченным рассмотрим $e_n = (0, \ldots, 0, 1, 0, \ldots)$, то есть все $0$, кроме $1$ на $n$-ом месте. Тогда $\rho(e_n, e_m) = 1$, то есть мы получили бесконечную систему, между которыми попарные расстояния равны $1 \Rightarrow$ нет вполне ограниченности.
\end{example}
\begin{theorem}
    (Гейне-Борель 2.0) Пусть $(X, \rho)$~---~метрическое пространство, $K \subset X$~---~компакт. Тогда для любого открытого покрытия $\{ U_{\alpha}\}_{\alpha \in A}$ компакта существует конечное подпокрытие $\{ U_{\alpha_{i}}\}_{i = 1}^{N}$.
\end{theorem}
\begin{proof}
    Так как из компактности следует вполне ограниченность, то $\forall n \in \N$ $\exists$ конечная $\frac{1}{n}$-сеть, которую обозначим как $\{ z_{n} (1), \ldots, z_{n} (N)\}$.

    Предположим, что существует открытое покрытие $\{ U_{\alpha}\}_{\alpha \in A}$ из которого нельзя извлечь конечное подпокрытие. Тогда существует шар радиуса $\frac{1}{n}$ в центром в какой-то точке $\frac{1}{n}$-сети, то есть $\exists i \in \{ 1, \ldots, N\}$: $B_{\frac{1}{n}} (z_{n} (i))$ нельзя покрыть конечным набором множеств из системы $\{ U_{\alpha}\}_{\alpha \in A}$, так как в противном случае каждый шар $B_{\frac{1}{n}} (z_{n} (i))$ покрывался бы конечным числом элементом из покрытия $\{U_{\alpha}\}_{\alpha \in A}$, а так как шаров конечное число и они покрывают $K$, то получили бы конечное подпокрытие $K$, а мы предположили, что его нет.

    Получается, $\forall n \in \Z$ $\exists z_{n} \in K$: $B_{\frac{1}{n}} (z_{n})$ не может быть покрыт конечным числом элементов из $\{ U_{\alpha}\}_{\alpha \in A}$, получаем последовательность $\{ z_{n}\}$, из неё, так как $K$ компакт, можно извлечь сходящуюся подпоследовательность $\{ z_{n_{m}}\} \Rightarrow \exists z^{*} \in K$: $\rho(z_{n_{m}}, z^{*})\to 0$, $m\to \infty$. Так как $z^{*} \in K$, а $\{ U_{\alpha}\}_{\alpha \in A}$~---~покрытие, то $\exists \alpha^{*} \in A$: $z^{*} \in U_{\alpha^{*}}$, но $U_{\alpha^{*}}$ открытое множество $\Rightarrow \epsilon^{*} > 0$: $B_{\epsilon^{*}} (z^{*}) \subset U_{\alpha^{*}}$. 
    
    Так как $\{ z_{n_{m}}\}$ сходится к $z^{*}$, то начиная с некоторого номера $\rho(z_{n_m}, z^{*})$ можно сделать меньше, чем $\frac{\epsilon^*}{4}$. Запишем более формально: $\exists M \in \N$: $\forall m \geq M \hookrightarrow \rho(z_{n_m}, z^*) < \frac{\epsilon^{*}}{4}$ и $\frac{1}{n_m} < \frac{\epsilon^{*}}{4}$ по построению. Рассмотрим $\displaystyle B_{\frac{1}{n_m}} (z_{n_m})$. Заметим, что $\displaystyle B_{\frac{1}{n_m}} (z_{n_m}) \subset B_{\epsilon^{*}} (z^{*}) \subset U_{\alpha^{*}} \Rightarrow \forall m \geq M \hookrightarrow B_{\frac{1}{n_m}} (z_{n_m}) \subset U_{\alpha^{*}}$~---~противоречие с построением.

    \underline{Замечание.} Включение $\displaystyle B_{\frac{1}{n_m}} (z_{n_m}) \subset B_{\epsilon^{*}} (z^{*})$ верно по неравенству треугольника, то есть возьмём $\displaystyle y \in B_{\frac{1}{n_m}} (z_{n_m})$, тогда $\rho(z^{*}, y) \leq \rho(z^{*}, z_{n_m}) + \rho(z_{n_m}, y) < \frac{\epsilon^{*}}{4} + \frac{1}{n_m} < \frac{\epsilon^{*}}{4} + \frac{\epsilon^{*}}{4} < \frac{\epsilon^{*}}{2}$
\end{proof}
\begin{theorem}
    Пусть $(X, \rho)$~---~метрическое пространство. Если $K$~---~компакт, то он ограничен и замкнут.
\end{theorem}
\begin{proof}
    Ограниченность уже была доказана ранее. Докажем замкнутость. Будем доказывать от противного.

    Предположим противное, то есть $\exists x^{*} \not\in K$ являющаяся его точкой прикосновения. Тогда по критерию точки прикосновения $\exists \{ x_{n} \} \subset K$: $\rho(x^{*}, x_n)\to 0$, $n\to \infty \Rightarrow \forall \{ x_{n_{j}}\}$ подпоследовательности последовательности $\{ x_{n}\}$ тоже сходится к $x^{*} \not\in K$~---~противоречие с компактностью.
\end{proof}
\begin{reminder} На пространство $\R^{n}$ можно смотреть по-разному.

    $\R^{n} := \left\{ x = (x_{1}, \ldots, x_{n})\text{: }x_{i} \in \R \ \forall i \in \{ 1, \ldots, n\}\right\}$.
    
    $\displaystyle \rho(x, y) := \sqrt{\sum_{i = 1}^{n} (x_{i} - y_{i})^{2}}$. $\quad \displaystyle \| x \| = \sqrt{\sum_{i = 1}^{n} x_{i}^{2}}$.$\quad$ <$x$, $y$>$\displaystyle = \sum_{i = 1}^{n} x_{i} y_{i}$.
\end{reminder}
\begin{lemma}
    Последовательность $\{ x^m \} \subset \R^n$ сходится к $x^{*} \Leftrightarrow \forall i \in \{ 1, \ldots, n\}$ $x_{i}^{m}\to x^{*}_{i}$, $m\to \infty$.
\end{lemma}
\begin{proof}
    \underline{Шаг 1.} Заметим, что $\forall i \in \{ 1, \ldots, n\}$ справедливо неравенство

    $$\displaystyle |x_{i}^{*} - x_{i}^{m}| \leq \sqrt{|x_{i}^{*} - x_{i}^{m}|^{2}} \leq \sqrt{\sum_{i = 1}^{n}|x_{i}^{*} - x_{i}^{m}|^{2}} = \rho(x^{*}, x^{m}).$$

    \underline{Шаг 2.} Пусть $\forall i \in \{1, \ldots, n\} \hookrightarrow x^{m}_{i}\to x^{*}_{i},$ $m\to \infty$. Тогда $\displaystyle \sum_{i = 1}^{n} |x^{*}_{i} - x^{m}_{i}|^{2}\to 0$, $m\to \infty$. Извлекая корень, получаем

    $$\displaystyle \sqrt{\sum_{i = 1}^{n}|x_{i}^{*} - x_{i}^{m}|^{2}} \to 0, m\to \infty.$$
\end{proof}
\begin{theorem}
    \hypertarget{thm7.10}{(Теорема Больцано-Вейерштрасса в $\R^{n}$) Из любой ограниченной последовательности $\{ x^m \}_{m = 1}^{\infty}$ можно выделить сходящуюся подпоследовательность $\{ x^{m_{j}} \}_{j = 1}^{\infty}$.}
\end{theorem}
\begin{proof}
    Доказательство будем проводить по индукции (по размерности пространства).
    
    База: при $n = 1$ мы её \hyperlink{thm2.8}{уже доказали}.

    Предположим, что доказано при $n_{0} \in \N$, докажем для $n_{0} + 1$. Возьмём последовательность $x^{m} = (x^{m}_{1}, \ldots, x^{m}_{n_{0} + 1}) \subset \R^{n_{0} + 1}$~---~она ограничена. Спроектируем теперь точки $x^{m}$ на $\R^{n_{0}} \times \{0\}$, получим последовательность $\{\overline{x}^{m}\} \subset \R^{n_{0}}$, но эта последовательность тоже ограничена, так как $\displaystyle \| \overline{x}^{m}\| = \sqrt{\sum_{i = 1}^{n_{0}} \left(\overline{x}^{m}_{i}\right)^{2}} \leq \sqrt{\sum_{i = 1}^{n_{0} + 1} \left(\overline{x}^{m}_{i}\right)^{2}} \leq C$ $\forall m \in \N$.

    По предположению индукции Больцано-Вейерштрасс работает для размерности $n_{0}$, тогда $\exists \{\overline{x}^{m_{k}}\}$ сходящаяся к $\overline{x}^{*} \in \R^{n_{0}}$.

    Последовательность $\left\{ x^{m_{k}}_{n_{0} + 1}\right\}$ ограничена в $\R \Rightarrow$ по теореме Больцано-Вейерштрасса $\exists \left\{x^{m_{k_{j}}}_{n_{0} + 1}\right\}$: $\left\{x^{m_{k_{j}}}_{n_{0} + 1}\right\}\to x^{*}_{n_{0} + 1}$, $\j\to \infty$. Рассмотрим вектор $x^{*} = (\overline{x}^{*}, x^{*}_{n_{0} + 1})$. В итоге так как $\left\{ \overline{x}^{m_{k_j}}\right\}\to \overline{x}^{*}$, $j\to \infty$, получается, что $x^{m_{k_j}}\to x^{*}$, $j\to \infty$, то есть мы сделали шаг индукции.
\end{proof}
\begin{theorem}
    (Критерий компактности в $\R^n$) Множество $K \subset \R^n$~---~компакт $\Leftrightarrow$ оно ограничено и замкнуто.
\end{theorem}
\begin{proof}
    \underline{Шаг 1.} В одну сторону мы \hyperlink{lemm7.6}{уже доказали} в случае общих метрических пространств.

    \underline{Шаг 2.} Пусть $K$~---~ограничено и замкнуто. Возьмём произвольную последовательность $\{ x^{m} \} \subset K$. Поскольку $K$ ограничено, то по \hyperlink{thm7.10}{теореме Больцано-Вейерштрасса в $\R^{n}$} $\exists \{x^{m_{j}}\} \subset K$: она сходится к некоторой точке $x^{*} \in \R^n \Rightarrow$ в силу критерия точки прикосновения $x^{*}$~---~точка прикосновения для $K$, но в силу замкнутости $K$ получаем $x^{*} \in K$.
\end{proof}
\begin{theorem}
    (Критерий Коши в $\R^n$) Последовательность $\{ x^{m}\} \subset \R^{n}$ сходится $\Leftrightarrow$ когда выполнено условие Коши, то есть $\displaystyle \forall \epsilon > 0 \ \exists N (\epsilon)$: $\displaystyle \forall m, l \geq N (\epsilon) \hookrightarrow \sqrt{\sum_{i = 1}^{n} (x^{m}_{i} - x^{l}_{i})^{2}} \leq \epsilon$.
\end{theorem}
\begin{proof}
    Точно такое же, как и в одномерном случае с учётом теоремы Больцано-Вейерштрасса.
\end{proof}

\section{Кривые}
\subsection{Вектор-функции}
\begin{definition}
    Отображение $\E \subset \R$ в $\R^n$ называется \textit{вектор-функцией}.
\end{definition}