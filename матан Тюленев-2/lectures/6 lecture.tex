\begin{lemma}
    \hypertarget{lemm2.8}{Если последовательность} $\{ x_{n} \}$~---~неограничена сверху, то $+ \infty$ является её частичным пределом. Если $\{ x_{n} \}$ неограничена снизу, то $- \infty$ является её частичным пределом.
\end{lemma}
\begin{proof}
    Докажем для случая неограниченности сверху, так как случай неограниченности снизу рассматривается аналогично.

    Заметим, что если $\{ x_{n} \}$ неограничена сверху, то $\forall N \in \N$ отбросим первые $N$ элементов и снова получим последовательность, неограниченную сверху. Рассмотрим последовательность $\{ y_{n} \} = \{ x_{n + N}\} \Rightarrow \forall \epsilon > 0 \  n \in \N$: $y_{n} > \frac{1}{\epsilon} \Rightarrow \forall N \in \N \  \forall \epsilon > 0 \  \exists k > N$: $x_{k} > \frac{1}{\epsilon} \Rightarrow$ по критерию частичного предела $+ \infty$ является частичным пределом последовательности $\{ x_{n} \}$.
\end{proof}
\begin{definition}
    $PL (\{ x_{n} \}) := \{ L \in \R$: $L$~---~частичный предел $\{ x_{n} \} \}$~---~\textit{множество всех частичных пределов}.
\end{definition}
\begin{theorem}
    (Обобщённая теорема Больцано-Вейерштрасса) Любая числовая последовательность имеет хотя бы один частичный предел в $\overline{\R}$.
\end{theorem}
\begin{proof}
    Доказательство состоит в применении критерия частичного предела и \hyperlink{lemm2.8}{леммы 2.8}.
\end{proof}
\begin{definition}
    Пусть $A \subset \overline{\R}$. $M \in \overline{\R}$~---~супремум $A$.
    $$M = \sup A \Leftrightarrow \begin{cases}
        a \leq M, \  \forall a \in A; \\
        \forall M^{'} < M \  \exists a \in A: M^{'} < a.
    \end{cases}$$
\end{definition}
\begin{definition}
    Пусть $A \subset \overline{\R}$. $m \in \overline{\R}$~---~инфимум $A$. 
    $$m = \inf A \Leftrightarrow \begin{cases}
        a \geq m, \  \forall a \in A; \\
        \forall m^{'} > m \  \exists a \in A: m^{'} > a.
    \end{cases}$$
\end{definition}
\begin{definition}
    \textit{Верхним и нижним пределом} последовательности $\{ x_{n} \}$ называются соответственно
    $$ \overline{\lim\limits_{n\to \infty}} x_{n} = \sup PL (\{ x_{n} \});$$
    $$ \lim\limits_{\overline{n\to \infty}} x_{n} = \inf PL (\{ x_{n} \}).$$
\end{definition}
\begin{lemma}
    \hypertarget{lemm2.9}{Пусть} последовательность $\{ x_{n} \}$ имеет предел равный $A$, $A \in \overline{\R}$. Тогда $PL (\{ x_{n} \}) = \{A \}$.
\end{lemma}
\begin{proof}
    Запишем определение предела:
    $$ \forall \epsilon > 0 \  \exists N (\epsilon) \in \N: \forall n \geq N \hookrightarrow x_{n} \in U_{\epsilon} (A).$$

    Возьмём произвольную подпоследовательность $x_{n_{k}}$ последовательности $\{ x_{n} \}$ и покажем, что $\lim\limits_{k\to \infty} x_{n_{k}} = A$.

    Из того, что $n_{k}$ строго возрастает легко доказать, что $n_{k} \geq k \Rightarrow \forall \epsilon > 0 \  \exists K (\epsilon) = N (\epsilon)$: $\forall k \geq K (\epsilon) (\Rightarrow n_{k} \geq N (\epsilon)) \hookrightarrow x_{n_{k}} \in U_{\epsilon} (A)$.
\end{proof}
\begin{theorem}
    \hypertarget{thm2.10}{Пусть дана числовая последовательность} $\{ x_{n} \}$. Тогда $\overline{\lim\limits_{n\to \infty}} x_{n} \in PL (\{ x_{n}\})$ и $\lim\limits_{\overline{n\to \infty}} x_{n} \in PL (\{ x_{n} \})$.
\end{theorem}
\begin{proof}
    Докажем для верхнего предела, а для нижнего аналогично.

    Обозначим $M = \sup PL (\{ x_{n} \})$. Из определения супремума следует, что $\forall \epsilon^{*} > 0 \hookrightarrow U_{\epsilon / 2} (M) \cap PL (\{ x_{n} \}) \neq \varnothing.$ Следовательно, $\exists c \in PL (\{ x_{n} \})$: $c \in U_{\epsilon / 2} (M) \Rightarrow$ по критерию частичного предела в $U_{\epsilon / 2} (c)$ содержатся значения бесконечного количества элементов последовательности $\{ x_{n} \}$.

    $U_{\epsilon / 2} (c) \subset U_{\epsilon} (M)$, так как $c \in U_{\epsilon} (M)$. Тогда в $U_{\epsilon} (M)$ содержатся значения бесконечного количества элементов последовательности $\{ x_{n} \}$, но $\epsilon$ был выбран произвольно $\Rightarrow \forall \epsilon > 0$ в $U_{\epsilon} (M)$ содержатся значения бесконечного количества элементов последовательности $\{ x_{n} \} \Rightarrow$ по критерию частичного предела $M$~---~частичный предел.
\end{proof}
\begin{theorem}
    \hypertarget{thm2.11}{Пусть} $\{ x_{n} \}$~---~числовая последовательность. Тогда для верхнего предела справедливо равенство
    $$ \overline{\lim\limits_{n\to \infty}} x_{n} = \underset{n \in \N}{\inf} ({\sup\limits_{k \geq n}} \  x_{k}),$$
    а для нижнего предела
    $$ \lim\limits_{\overline{n\to \infty}} x_{n} = \underset{n \in \N}{\sup} ({\inf\limits_{k\geq n}} \  x_{k}).$$
\end{theorem}
\begin{proof}
    Докажем для верхнего предела, а для нижнего аналогично. 

    \underline{Шаг 0.} При каждом $n \in \N$ положим
    $$ y_{n} := \sup\limits_{k \geq n} x_{k} := \sup \{ x_{k}: k \geq n\}.$$
    Заметим, что $y_{n^{*}} = + \infty$ при каком-то $n^{*} \in \N \Leftrightarrow y_{n} = + \infty \  \forall n \in \N$. Действительно, это легко следует из того, что хвосты $\{ x_{k}$: $k \geq n_{1}\}$ и $\{ x_{k}$: $k \geq n_{2}\}$ отличаются не более, чем на конечное число элементов. Если $y_{n} = + \infty$ при всех $n \in \N$, то, очевидно, $\overline{\lim\limits_{n\to \infty}} x_{n} = + \infty.$ С другой стороны, из определения супремума и равенства $y_{n} = + \infty$ при всех $n \in \N$ вытекает, что $\forall \epsilon > 0 \  \forall n \in \N \  \exists k \geq n$ такой, что $x_{k} > \frac{1}{\epsilon}$. По критерию частичного предела это означает, что $+\infty$ является частичным пределом последовательности $\{ x_{n} \}$, а значит $\overline{\lim\limits_{n\to \infty}} x_{n} = + \infty.$ Резюмируя, получаем, что, если $\exists n \in \N$: $y_{n} = + \infty$, то требуемое равенство выполнено.

    \underline{Шаг 1.} Заметим, что $y_{n} \geq x_{n} > -\infty \  \forall n \in \N$. Так как случай $y_{n} = +\infty$ уже рассмотрен выше, \textit{до конца доказательства предполагаем, что} $\{ y_{n} \} \subset \R$. Ключевым же для нас наблюдением будет монотонность последовательности $\{ y_{n} \}$. Действительно, из включения
    $$ \{ k \in \N: k \geq n + 1 \} \subset \{ k \in \N: k \geq n \}$$
    вытекает неравенство
    $$ y_{n + 1} \leq y_{n} \  \forall n \in \N. \qquad (1)$$

    \underline{\hypertarget{step2}{Шаг 2.}} Установим неравенство
    $$ \overline{\lim\limits_{n\to \infty}} x_{n} \leq \underset{n \in \N}{\inf} ({\sup\limits_{k \geq n}} \  x_{k}).$$

    \hyperlink{thm2.10}{По теореме 2.10} верхний предел сам является частичным пределом. Фиксируем произвольную подпоследовательность $\{ x_{n_{j}} \}$ последовательности $\{ x_{n} \}$, для которой
    $$ \overline{\lim\limits_{n\to \infty}} x_{n} = \lim\limits_{j\to \infty} x_{n_{j}}. \qquad (2)$$

    Из положенного нами определения $y_{n}$ (в шаге 0) очевидно, что
    $$ x_{n_{j}} \leq y_{n_{j}}.$$

    Поскольку в силу неравенства (1) последовательность $\{ y_{n} \}$ монотонно убывает (нестрого), она имеет предел. А значит имеет предел и подпоследовательность $\{ y_{n_{j}}\}$. Более того,
    $$ \lim\limits_{j\to \infty} y_{n_{j}} = \lim\limits_{n\to \infty} y_{n}. \quad (3)$$

    Переходя к пределу в неравенстве $x_{n_{j}} \leq y_{n_{j}}$ и используя (2) и (3), получим
    $$ \overline{\lim\limits_{n\to \infty}} x_{n} \leq \lim\limits_{j\to \infty} y_{n_{j}} = \lim\limits_{n\to \infty} y_{n} = \inf\limits_{n \in \N} y_{n}. $$

    Из положенного нами определения $y_{n}$ и неравенства выше и вытекает требуемое в начале шага 2 неравенство.

    \underline{Шаг 3.} Установим неравенство
    $$ \overline{\lim\limits_{n\to \infty}} x_{n} \geq \underset{n \in \N}{\inf} ({\sup\limits_{k \geq n}} \  x_{k}).$$

    Покажем, что правая часть этого неравенства является частичным пределом. Тогда, учитывая, что верхний предел - это наибольший (в смысле $\overline{\R}$) частичный предел, получаем требуемое неравенство.

    Обозначим правую часть неравенства за $M$. Тогда в силу монотонности $\{ y_{n} \}$ и теоремы Вейерштрасса получим
    $$ \forall \epsilon > 0 \  \exists N (\epsilon) \in \N: \forall n \geq N (\epsilon) \hookrightarrow y_{n} \in U_{\epsilon / 2} (M). \quad (4)$$

    Используя определение супремума и наше определение $y_{n}$, получим, что
    $$ \forall \epsilon > 0 \  \forall n \geq N (\epsilon) \  \exists k \geq n: x_{k} \in U_{\epsilon / 2} (y_{n}). \quad (5)$$

    Заметим, что, если $y_{n} \in U_{\epsilon / 2} (M)$ и $x_{k} \in U_{\epsilon / 2} (y_{n})$, то $x_{k} \in U_{\epsilon} (M)$. Следовательно, учитывая (4) и (5), получаем, что $\forall \epsilon > 0 \  U_{\epsilon} (M)$ содержит значения бесконечного числа элементов последовательности $\{ x_{n} \}$. По критерию частичного предела это означает, что $M$~---~частичный предел. Тогда требуемое неравенство доказано.

    \underline{Шаг 4.} Объединяя доказанные в шагах 2 и 3 неравенства, получаем требуемое равенство.
\end{proof}


\newpage
    \begin{tikzpicture}
    % Рисуем сетку
    \draw[help lines, step=0.25, dotted]
    (-4.5,-5) grid (7,5);
    % Начало координат
    \draw[->, thin] (-4.5,-5) -- (7.05,-5);
    \node[below, scale = 0.75] at (7.7,-5) {$\quad n, \  n\in \N$}; % Ox
    \draw[->, thin] (-4.5,-5) -- (-4.5,5.05)
    node[left, scale = 0.75] {$x_{n}, x_{n} \in \R$}; % Oy

    \draw[dotted, line width =.02cm] (0, -5) -- (0, 5);
    \node[right] at (0,3) {$\textrm{Отсечём <<хвост>>}$};

    \draw[fill = black] (-4, -1) circle (1.25pt);
    \draw[fill = black] (-3, 0.5) circle (1.25pt);
    \draw[fill = black] (-2, 4.5) circle (1.25pt);
    \draw[fill = black] (-1, 3) circle (1.25pt);
    \draw[fill = black] (0, 0) circle (1.25pt);
    \draw[fill = black] (1, -1.5) circle (1.25pt);
    \draw[fill = black] (2, 1) circle (1.25pt);
    \draw[fill = black] (3, -4) circle (1.25pt);
    \draw[fill = black] (4, -0.5) circle (1.25pt);
    \draw[fill = black] (5, -3) circle (1.25pt);
    \draw[fill = black] (6, -1.5) circle (1.25pt);

    \draw[red, fill = red] (-4, 4.5) circle (2pt);
    \draw[red, fill = red] (-3, 4.5) circle (2pt);
    \draw[red] (-2, 4.5) circle (2pt);
    \draw[red] (-1, 3) circle (2pt);
    \draw[red, fill = red] (0, 1) circle (2pt);
    \draw[red, fill = red] (1, 1) circle (2pt);
    \draw[red]  (2, 1) circle (2pt);
    \draw[red, fill = red] (3, -0.5) circle (2pt);
    \draw[red]  (4, -0.5) circle (2pt);
    \draw[red, fill = red] (5, -1.5) circle (2pt);
    \draw[red]  (6, -1.5) circle (2pt);



    \node[below, scale =0.65] at (-4, -5.1) {$1$};
    \node[below, scale =0.65] at (-3, -5.1) {$2$};
    \node[below, scale =0.65] at (-2, -5.1) {$3$};
    \node[below, scale =0.65] at (-1, -5.1) {$4$};
    \node[below, scale =0.65] at (-0.5, -5.3) {$\ldots$};
    \node[below, scale =0.65] at (-0, -5.18) {$m$};
    \node[below, scale =0.65] at (1, -5.1) {$m+1$};
    \node[below, scale =0.65] at (2, -5.1) {$m+2$};
    \node[below, scale =0.65] at (3, -5.1) {$m+3$};
    \node[below, scale =0.65] at (4, -5.1) {$m+4$};
    \node[below, scale =0.65] at (5, -5.1) {$m+5$};
    \node[below, scale =0.65] at (6, -5.1) {$m+6$};
    \node[below, scale =0.65] at (6.75, -5.3) {$\ldots$};
        
    \draw[line width =.05cm, red] (4.25, 4) -- (5.25, 4);
    \draw[line width =.05cm ] (4.25, 3) -- (5.25, 3);
    \node[red] at (6, 4) {$\sup\limits_{k \geq n} x_{k}$};
    \node at (5.7, 3) {$x_{n}$};
    \end{tikzpicture}


\newpage

\begin{theorem}
    Пусть $\{ x_{n} \}$~---~числовая последовательность. Пусть $A \in \overline{\R}$. Следующие условия эквивалентны:
    \begin{enumerate}
        \item $\exists \lim\limits_{n\to \infty} x_{n} = A;$
        \item $PL (\{ x_{n} \}) = \{ A \};$
        \item $ \lim\limits_{\overline{n\to \infty}} x_{n} = \overline{\lim\limits_{n\to \infty}} x_{n} = A.$
    \end{enumerate}
\end{theorem}
\begin{proof}
    Заметим, что $(2) \Leftrightarrow (3)$ просто из определений. Покажем $(3) \to (2)$. Пусть $c \in PL (\{ x_{n} \})$.
    $$ \lim\limits_{\overline{n\to \infty}} x_{n} \leq c \leq \overline{\lim\limits_{n\to \infty}} x_{n}, \  \text{но они равны по (3)} \Rightarrow c = A. $$ Переход $(2) \to (3)$ очевиден. Множество $PL$ состоит всего из одного элемента, то есть крайне легко взять его супремум и инфимум (и они равны).

    Переход $(1) \to (2)$ доказали ранее (\hyperlink{lemm2.9}{лемма 2.9}).

    Покажем $(3) \to (1)$, тогда мы докажем равносительность всех условий.

    Сначала рассмотрим случай $A \in \R$. Тогда, используя рассуждения \hyperlink{step2}{шага 2} теоремы 2.11, легко видеть, что $z_{n} := \inf \{ x_{k}$: $k \geq n\} \in \R$ и $y_{n} := \sup \{ x_{k}$: $k \geq n\} \in \R$ при всех $n \in \N$. Используя монотонность последовательностей $\{ z_{n} \}$, $\{ y_{n} \}$ и учитывая \hyperlink{thm2.10}{теорему 2.10}, получаем, что
    $$ \lim\limits_{\overline{n\to \infty}} x_{n} = \lim\limits_{n\to \infty} z_{n} = \overline{\lim\limits_{n\to \infty}} x_{n} = \lim\limits_{n\to \infty} y_{n} = A.$$
    Тогда, используя очевидное неравенство $z_{n} \leq x_{n} \leq y_{n}$, справедливое при всех $n \in \N$, и \hyperlink{thm2.4}{теорему о трёх последовательностях}, получаем, что $\exists \lim\limits_{n\to \infty} x_{n} = A$.
    
    Рассмотрим случай $A = +\infty$, так как случай $A = -\infty$ рассматривается аналогично. Если $\lim\limits_{\overline{n\to \infty}} x_{n} = +\infty$, то \hyperlink{thm2.11}{по теореме 2.11} в силу определения инфимума получаем, что $\forall \epsilon > 0 \  \exists N (\epsilon) \in \N$: $z_{N (\epsilon)} \in U_{\epsilon} (+ \infty)$. По построению последовательности $z_{n}$ получаем, что $\forall \epsilon > 0 \  \exists N (\epsilon) \in \N$: $\forall n \geq N (\epsilon) \hookrightarrow x_{n} \in U_{\epsilon} (+ \infty)$, а значит $\exists \lim\limits_{n\to \infty} x_{n} = +\infty$.
\end{proof}

\subsection{Критерий Коши}
\begin{definition}
    Последовательность $\{ x_{n} \}$ называется \textit{фундаментальной}, если выполнено \textit{условие Коши}:
    $$ \forall \epsilon > 0 \  \exists N (\epsilon) \in \N: \forall n \geq N, \forall m \geq N \hookrightarrow |x_{n} - x_{m}| < \epsilon. $$
\end{definition}
\begin{lemma}
    \hypertarget{lemm2.10}{Пусть} $\{ x_{n} \}$~---~сходящаяся последовательность. Тогда она фундаментальна.
\end{lemma}
\begin{proof}
    Пусть $\{ x_{n} \}$~---~ сходится к числу $c$. Тогда
    $$ \forall \epsilon > 0 \  \exists N (\epsilon) \in \N: \forall n \geq N (\epsilon) \hookrightarrow |x_{n} - c| < \epsilon / 2$$
    $$ \text{следовательно, }\  \forall \epsilon > 0  \  \exists N (\epsilon): \forall n \geq N \  \forall m \geq N \hookrightarrow |x_{n} - x_{m}| \leq |x_{n} - c| + |x_{m} - c| < \frac{\epsilon}{2} + \frac{\epsilon}{2} = \epsilon.$$
\end{proof}
\begin{lemma}
    \hypertarget{lemm2.11}{Если последовательность} $\{ x_{n} \}$ фундаментальна, то она ограничена.
\end{lemma}
\begin{proof}
    Поскольку условие фундаментальности сформулировано для любого $\epsilon$, то возьмём $\epsilon = 1$.
    $$ \exists N (1): \forall n, m \geq N (1) \hookrightarrow |x_{n} - x_{m}| < 1, \text{ возьмём $m = N (1)$} \Rightarrow$$
    $$ \Rightarrow |x_{n} - x_{N (1)}| < 1 \  \forall n \geq N (1) \Rightarrow |x_{n}| < 1 + |x_{N (1)}| \  \forall n \geq N (1)$$

    Положим $M := max \{ x_{1}, x_{2}, \ldots, |x_{N (1)}|, 1 + |x_{N (1)}|\} \Rightarrow |x_{n}| \leq M \   \forall n \in \N \Rightarrow \{ x_{n} \}$~---~ограниченная последовательность.
\end{proof}
\begin{theorem}
    (Критерий Коши) Последовательность сходится тогда и только тогда, когда она фундаментальна.
\end{theorem}
\begin{proof}
    В одну сторону мы уже доказали (\hyperlink{lemm2.10}{лемма 2.10}). Докажем в другую сторону.

    Пусть последовательность $\{ x_{n} \}$ фундаментальна $\Rightarrow$ она ограничена \hyperlink{lemm2.11}{по лемме 2.11}, а тогда у $\{ x_{n} \}$ есть конечный частичный предел \hyperlink{thm2.8}{по теореме Больцано-Вейерштрасса} $\Rightarrow \exists c \in \R$ и $\exists \{ x_{n_{k}} \}$: $\lim\limits_{k\to \infty} x_{n_{k}} = c \Rightarrow \exists k \in \N$: $n_{k} \geq N (\epsilon)$ и $|x_{n_{k}} - c| < \epsilon / 2$.
    $$ \forall \epsilon > 0 \  \exists N (\epsilon) \in \N: \forall n, m \geq N (\epsilon) \hookrightarrow |x_{n} - x_{m}| < \epsilon / 2 \Rightarrow$$
    $$ \Rightarrow \forall \epsilon > 0 \  \exists N (\epsilon) \in \N: \forall n \geq N (\epsilon) \hookrightarrow |x_{n} - c| \leq |x_{n} - x_{n_{k}}| + |x_{n_{k}} - c| \Rightarrow |x_{n} - c| < \epsilon.$$

    Теперь мы доказали утверждение в обе стороны.
\end{proof}