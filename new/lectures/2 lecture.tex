\begin{definition}
        Число $M (m)$ называется \textit{верхней (нижней) гранью} числового непустого множества $A \subset \R,$ если $x \leq M$ ($x \geq m$) $\quad \forall x \in A.$
    \end{definition}

    \begin{definition}
    \hypertarget{def1.23}{}
        Пусть $A$~---~ограниченное сверху множество. Число $M \in \R$ называется супремумом $A$ и записывается $M = sup A$, если выполняется:

        \begin{enumerate}
            \item $M$ является верхней гранью, то есть $\forall x \in A \hookrightarrow x \leq M$.
            \item $\forall M^{'} < M \quad \exists a(M^{'}) \in A$: $M^{'} < a(M^{'}) \leq M.$ То есть никакое другое число не является верхней гранью.
        \end{enumerate}
    \end{definition}

    \begin{definition}
        Если $A$~---~неограниченное сверху множество, то $sup A := +\infty .$
    \end{definition}

    \begin{theorem}{(о существовании и единственности супремума)} Супремум существует и единственен. $$\forall A \subset \R: A \neq \varnothing \hookrightarrow \exists! \sup{A}.$$
    \end{theorem}

    \begin{proof}
        В случае неограниченного множества $A$ верность теоремы следует из определения.
        Рассмотрим случай ограниченного множества $A \Rightarrow$ существует хотя бы одна верхняя грань.

        Пусть $B := \{M \in \R: M - \text{верняя грань } A\}. \quad B \neq \varnothing .$

        Кроме того $A$ расположенно левее $B$. Тогда в силу аксиомы непрерывности $\exists c \in \R$: $a \leq c \leq M \quad \forall a \in A, \quad \forall M \in B.$

        Покажем, что $c = \sup{A}.$ Действительно, так как $a \leq c \ \forall a \in A \Rightarrow c$~---~верхняя грань, тогда 1 пункт определния супремума проверен.

        Предположим $\exists c^{'} < c$: $c^{'}$~---~верхняя грань. Тогда $c^{'} \in B$, но $c$ было выбранно так, что $c \leq M \quad \forall M \in B \Rightarrow c \leq c^{'}$~---~противоречие $\Rightarrow \forall c^{'} < c \hookrightarrow c \notin B \Leftrightarrow \lnot (c^{'} \in B) \Leftrightarrow \lnot (\forall a \in A \hookrightarrow a \leq c^{'}) \Leftrightarrow \exists a(c^{'}) \in A: a(c^{'}) > c^{'}$, но так как $a(c^{'}) \in A$, то $a(c^{'}) \leq c.$ И тогда мы показали, что $\forall c^{'} < c \quad \exists a(c^{'}) \in A$: $c^{'} < a(c^{'}) \leq c.$ Значит, мы проверили определние супремума с заменой $M$ на $c \Rightarrow$ он существует. 

        Докажем единственность супремума. Предположим, что $\exists M_{1}, M_{2} \in \R$: $M_{1} = \sup{A}$ и $M_{2} = \sup{A}.$

        Пусть $M_{1} > M_{2}.$ Тогда по (2) пункту определения супремума (для $M_{1}$) $\exists a(M_{2}) \in A$: $a(M_{2}) > M_{2} \Rightarrow$ это противоречит тому, что $M_{2}$~---~верхняя грань (то есть (1) пункт определения $M_{2}$ как супремума) $\Rightarrow$ такого быть не может.

        Случай $M_{2} > M_{1}$ аналогичен $\Rightarrow M_{1} = M_{2},$ то есть супремум существует и единственнен.
    \end{proof}
    
    \newpage
    \begin{proposition}
        $M = \sup{A} \text{  } (M \in \overline{\R}, \  A \subset \R, \  A \neq \varnothing$) тогда и только тогда, когда

        \begin{equation*}
            \begin{cases}
                a \leq M \quad \forall a \in A\\
                \forall M^{'} < M \  \exists a(M^{'}) \in A: M^{'} < a(M^{'}) \leq M& 
            \end{cases}
        \end{equation*}

        Для случая $A$~---~ограниченное множество, это просто определние супремума. 
        
        Пусть $A$~---~неограниченно сверху, тогда $+\infty = \sup{A}$, но тогда система выше выполняется при замене $M$ на $+\infty$ по отношению порядка $\overline{\R}.$

        И наоборот если система выполнена для $M = +\infty$, то тогда $A$~---~неограниченно сверху, и тогда $+\infty = \sup{A}.$
    \end{proposition}

    \begin{lemma}
        (Лемма Архимеда) Множество натуральных чисел неограниченно сверху. $$\forall M^{'} \in \R \text{  } \exists N (M^{'}) \in \N: N (M^{'}) > M^{'}.$$
    \end{lemma}
    \begin{proof}
        Предположим, что $\N$~---~ограниченно сверху $\Rightarrow$ существует верхняя грань и более того существует конечный супремум $M = \sup{\N} < +\infty.$ Тогда в силу второго пункта определения супрерума: $\forall M^{'} < M$ найдётся натуральное число его больше. Но так как это верно $\forall M^{'}$, то можем взять $M^{'} = M - 1. \newline$ Тогда $\exists N(M^{'}) \in \N$: $N(M^{'}) > M - 1 \Rightarrow N(M^{'}) + 1 > M \Rightarrow M$~---~не супремум. Противоречие.
    \end{proof}

    \begin{definition}
    \hypertarget{def1.25}{}
        $m \in \R$ называется \textit{инфимумом ограниченного снизу множества} $A$, если

        \begin{equation*}
            m = \inf{A} \Longleftrightarrow
            \begin{cases}
                a \geq m \quad \forall a \in A\\
                \forall m^{'} > m, \  \exists a(m^{'}) \in A: m^{'} > a(m^{'}) \geq m& 
            \end{cases}
        \end{equation*}
    \end{definition}

    \begin{definition}
        Если $A$~---~неограниченное снизу множество, то $\inf{A} := -\infty .$
    \end{definition}

    \begin{theorem}{(о существовании и единственности инфимума)} Инфимум существует и единственнен. $$\forall A \subset \R: A \neq \varnothing \hookrightarrow \exists! \inf{A}.$$
    \end{theorem}

    \begin{proof}
        Аналогично супремуму с точностью до замены знаков.
    \end{proof}

    \begin{proposition}
        $m = \inf{A} \text{  } (m \in \overline{\R}, \  A \subset \R, \  A \neq \varnothing$) тогда и только тогда, когда

        \begin{equation*}
            \begin{cases}
                a \geq m \quad \forall a \in A\\
                \forall m^{'} > m \  \exists a(m^{'}) \in A: m^{'} > a(m^{'}) \geq m& 
            \end{cases}
        \end{equation*}
    \end{proposition}

    \begin{definition}
        Число $M$ называется \textit{максимумом (максимальным элементом)} множества $E \subset \R \Leftrightarrow M = max E$, если
        \begin{enumerate}
            \item $M \in E$;
            \item $M \geq x \  \forall x \in E.$
        \end{enumerate}
        Аналогично определяется минимум.
    \end{definition}

    \newpage
    \subsection{Вложенные отрезки}
    Всегда предплогается, что $a_{n} \leq b_{n}.$
    \begin{definition}
        Отображение из $\N$ в множество всех отрезков на числовой прямой $\R$ назовём \textit{последовательностью отрезков} и обозначим $\{ [a_{n}, b_{n}] \}^{\infty}_{n = 1}$
    \end{definition}

    \begin{definition}
        Будем говорить, что $\{ [a_{n}, b_{n}] \}^{\infty}_{n = 1}$~---~последовательность \textit{вложенных отрезков}, если $\{ [a_{n+1}, b_{n+1}] \} \subset \{ [a_{n}, b_{n}] \} \quad \forall n \in \N$
    \end{definition}
    \begin{lemma}
        (Лемма Кантора или принцип вложенных отрезков) Любая последовательность вложенных отрезков имеет непустое пересечение (точка лежит сразу во всех отрезках), то есть
        $$\forall \text{ вложенной } \{ [a_{n}, b_{n}] \}^{\infty}_{n = 1} \quad \exists x \in \bigcap_{n = 1}^{\infty} [a_{n}, b_{n}] \Longleftrightarrow \bigcap_{n = 1}^{\infty} [a_{n}, b_{n}] \neq \varnothing$$
    \end{lemma}
    \begin{proof}
        $\forall n \in \N$ справедливы неравенства:
        $$-\infty < a_{n} \leq a_{n+1} \leq b_{n+1} \leq b_{n} < +\infty.$$

        Заметим следующий факт (*):
        $$ \forall n, m \in \N \hookrightarrow -\infty < a_{n} \leq b_{m} < +\infty$$

        Действительно, предположим $m \geq n \Rightarrow$ по индукции $b_{m} \leq b_{n} \Rightarrow a_{m} \leq b_{m} \leq b_{n}.$
        
        Если же $m < n$, то $a_{m} \leq a_{n} \leq b_{n}.$

        $A := \{a_{1}, a_{2} \dots , a_{n}, \dots \}$~---~ множество «левых» концов.
        
        $B := \{b_{1}, b_{2} \dots , b_{m}, \dots \}$~---~множество «правых» концов.

        Из (*) получаем, что $A$ расположенно «левее» $B \Rightarrow \exists c \in \R$: $a_{n} \leq c \leq b_{m} \  \forall n, m \in \N \Rightarrow \newline \Rightarrow a_{n} \leq c \leq b_{n} \quad \forall n \in \N \Rightarrow c \in [a_{n}, b_{n}] \  \forall n \in \N \Rightarrow c \in \displaystyle \bigcap_{n = 1}^{\infty} [a_{n}, b_{n}]$
    \end{proof}
    \begin{note}
        Лемма Кантора о вложенных отрезках может не работать для интервалов.

        Пример: $\displaystyle a_{n} = 0 \  \forall n \in \N$, $\displaystyle b_{n} = \frac{1}{n} \  \forall n \in \N.$

        ($\displaystyle a_{n}, b_{n}) = \Big(0, \frac{1}{n} \Big).$ $\quad \displaystyle \bigcap_{n = 1}^{\infty} [a_{n}, b_{n}] = \varnothing.$

        Действительно, предположим $\displaystyle \exists x > 0$: $\displaystyle x \in \bigcap_{n = 1}^{\infty} \bigg( 0, \frac{1}{n} \bigg) \Rightarrow 0 < x < \frac{1}{n} \  \forall n \in \N \Rightarrow n < \frac{1}{x}$~---~противоречие c леммой Архимеда.
    \end{note}
    \begin{definition}
        Последовательность вложенных отрезков $\displaystyle \bigcap_{n = 1}^{\infty} [a_{n}, b_{n}]$ называется \newline \textit{стягивающейся}, если $\forall n \in \N \  \exists [a_{m(n)}, b_{m(n)}]$: $l \displaystyle < \frac{1}{n}$, где $l = (b_{i} - a_{i})$. $l$~---~длина.
    \end{definition}
    \begin{theorem}
        Стягивающаяся последовательность вложенных отрезков $\{ [a_{n}, b_{n}] \}^{\infty}_{n = 1}$ имеет единственную общую точку, то есть
        $$ \exists ! x \in \bigcap_{n = 1}^{\infty} [a_{n}, b_{n}]$$
    \end{theorem}
    \begin{proof}
        Ранее было доказано, что пересечение не пусто \bigg($\displaystyle \exists x \in \bigcap_{n = 1}^{\infty} [a_{n}, b_{n}]$\bigg).

        Тогда предположим, что $\displaystyle \exists x_{1}, x_{2} \in \bigcap_{n = 1}^{\infty} [a_{n}, b_{n}] \quad (x_{1} \neq x_{2}).$

        Так как $x_{1} \neq x_{2} \Rightarrow |x_{1} - x_{2}| > 0.$ Пусть $\displaystyle |x_{1} - x_{2}| = \frac{1}{M}.$ Но тогда по лемма Архимеда $\displaystyle \exists N \in \N$: $\displaystyle N > M \Rightarrow \frac{1}{N} < |x_{1} - x_{2}| \Rightarrow$ в силу того, что система отрезков стягивающаяся, то $\exists [a_{m(N)}, b_{m(N)}]$ длина которого $\displaystyle < \frac{1}{N}$, но по предположению $x_{1}, x_{2}$ принадлежат всем отрезкам этой последовательности, в частности $\displaystyle x_{1}, x_{2} \in [a_{m(N)}, b_{m(N)}] \Rightarrow |x_{1} - x_{2}| < \frac{1}{N} \Rightarrow |x_{1} - x_{2}| < |x_{1} - x_{2}|$~---~ противоречие. Получается $x_{1} = x_{2}.$
    \end{proof}
    \begin{theorem}
        (3 принципа непрерывности числовой прямой) Следующие утверждения эквивалентны:
        \begin{enumerate}
            \item Аксиома непрерывности.
            \item Существование $\inf$ и $\sup$ у любого непустого множества.
            \item Лемма Кантора о непустоте пересечения вложенной системы и лемма Архимеда.
        \end{enumerate}
    \end{theorem}