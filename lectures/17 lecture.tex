\begin{theorem}
    \hypertarget{thrm5.7}{(Инвариантность формы первого дифференциала и неинвариантность формы высших дифференциалов)} Пусть функция $y = y(x)$ дифференцируема в точке $x_{0},$ а функция $z = z(x)$ дифференцируема в точке $y_{0} = y(x_{0}).$

    Тогда дифференциал $z,$ рассматриваемый как функция лишь от $y$ в точке $y_{0},$  и дифференциал функции $z= z(y(x)) = f(x)$ в точке $x_{0}$ записываются одинаково. А именно $dz = z'(y_{0}) \ dy$. При этом в первом случае (когда $z=z(y)$) $dy = y-y_{0}$, а во втором $dy$~---~ дифференциал функции $y(x)$ в точке $x_{0}$
\end{theorem}
\begin{proof}
    Для функции $z = z(y)$ по определению дифференциала 
    
    $$dz(y_{0}) = z'(y_{0}) \ dy, \ dy = y-y_{0}$$

    Рассмотрим композицию функций $z = z(y(x)) = f(x)$.

    $$dz = f'(x_{0}) \ dx = z'(y(x_{0})) \ y'(x_{0}) \ dx = z'(y(x_{0})) \ dy(x_{0})$$
\end{proof}

\begin{note}
    Для второго дифференциала форма записи не инвариантна.
\end{note}
\begin{proof}
    Действительно, пусть функция $z = z(y)$ дважды дифференцируема в точке $y_{0}$

    $$d^{2} = z''(y_{0}) \ (dy)^{2}, \ dy = y-y_{0}$$

    Если же $z = z(y(x))$

    $$d^{2} z = f''(x_{0}) \ (dx)^{2} = \Bigl( z'(y(x)) \cdot y'(x) \Bigr)' \ (dx)^{2}, \Big|_{x = x_{0}}
    $$
    
    $$
    = \Bigl[z''(y(x)) \cdot (y'(x))^{2} + z'(y(x))\cdot y''(x_{0}) \Bigr] \cdot (dx)^{2} = z''(y_{0}) \cdot (y'(x_{0}))^{2} \cdot (dx)^{2} + z'(y(x_{0})) \cdot y''(x_{0}) \cdot (dx)^{2} =$$

    $$
    = z''(y_{0}) (dy)^2 + z'(y_{0}) \ d^{2} y(x_{0})
    $$
\end{proof}

\subsection{Формула Лейбница}
Введём некоторые обозначения: 
$$0! :=1$$
$$ n! := 1 \cdot 2 \cdot \ldots \cdot n, \quad n \in \N $$

$ \relax C_{n}^{k}$~---~ биномиальный коэффициент.
$$ \relax C_{n}^{k} := \cfrac{n!}{k!(n-k)!}$$

Соглашение: $u^{(0)}(x) \equiv u(x)$

\begin{theorem} \hypertarget{thrm5.8}{Формула Лейбница}
    Пусть $\exists u^{(n)}(x_{0}) \in \R $ и $\exists v^{(n)}(x_{0}) \in \R$

    Тогда $\exists (u\cdot v)^{n} = \sum\limits_{s=0}^{n} \relax C_{n}^{s} u^{(s)}(x_{0}) \cdot v^{(n-s)}(x_{0})$
\end{theorem}
\begin{proof}
    Докажем по индукции.

    База индукции: при $n =1 $ верно (\hyperlink{thrm5.3}{обычное правило дифференцирование произведения}).

    Пусть доказано при некотором $k \in \N,$ установим при $k + 1$

    То есть мы доказали формулу при $k \in \N$:
    $$(u \cdot v)^{(k)} = \sum\limits_{s=0}^{k} \relax C_{k}^{s} u^{(s)}(x_{0}) \cdot v^{(k-s)}(x_{0}) $$

    $$
    (u \cdot v)^{(k+1)} = \Bigg(\sum\limits_{s=0}^{k} \relax C_{k}^{s} u^{(s)}(x) \cdot v^{(k-s)}(x) \Bigg)' \Bigg|_{x=x_{0}} = 
    $$
    $$ =\sum\limits_{s=0}^{k} \relax C_{k}^{s} u^{(s+1)}(x_{0}) \cdot v^{(k-s)}(x_{0}) + \sum\limits_{s=0}^{k} \relax C_{k}^{s} u^{(s)}(x_{0}) \cdot v^{(k-s + 1)}(x_{0}) = (*)
    $$
     Произведем замену индексов: $ s + 1 = j, \ s+ j - 1$
    $$
    (*) = \sum\limits_{j=1}^{k+1} \relax C_{k}^{j-1} u^{(j)}(x_{0}) \cdot v^{(k-j + 1)}(x_{0}) + \sum\limits_{j=0}^{k} \relax C_{k}^{j} u^{(j)}(x_{0}) \cdot v^{(k-j + 1)}(x_{0}) =
    $$

    $$
    =  \relax C_{k}^{k} u^{(k+1)}(x_{0}) \cdot v(x_{0}) + \relax C_{k}^{0} u(x_{0}) \cdot v^{(k+ 1)}(x_{0}) + \sum\limits_{j=1}^{k} \Bigl(\relax C_{k}^{j-1} + \relax C_{k}^{j} \Bigr) u^{(j)}(x_{0}) \cdot v^{(k+1 -j)}(x_{0}) = 
    $$

    Заметим, что
    
    $
    \relax C_{k}^{k} = \relax C_{k+1}^{k+1} = 1,$
    
    $\relax C_{k}^{0} = \relax C_{k + 1}^{0} = 1,
    $
    
    $
    \relax C_{k}^{j-1} +  \relax C_{k}^{j} = \cfrac{k!}{(j-1)!(k-j + 1)!} + \cfrac{k!}{(j)!(k-j)!} =
    $
    $$
    = \cfrac{k!}{(j)!(k-j+1)!} \cdot ( k -j + 1 + j) = \cfrac{(k+1)!}{(j)!(k+1-j)!} = C_{k+1}^{j}
    $$

    С учетом этого перепишем выражение $(*)$

    $$
    (u \cdot v)^{(k+1)} = (*) = \sum\limits_{j=0}^{k+1} \relax C_{k+1}^{j} u^{(j)}(x_{0}) \cdot v^{(k+1-j)}(x_{0})
    $$

    Шаг индукции доказан. Значит, формула верна при всех $n \in \N.$
\end{proof}